\section{Mathematical Stories}\footnote{All stories are quoted from \underline{The Heart of Mathematics} by Edward Burger and Michael Starbird.}

\newif\ifnudges
%\nudgestrue

\subsection{Damsel in Distress}

Long ago, knights in shining armor battled dragons and rescued damsels in distress on a daily basis. Although it is not often stress the many of the surviving stories of chivalry, frequently the rescue involved logical thinking and creative problem-solving, and often the damsel provided the solution. Here then is a typical knightly encounter.

Once upon a time, a damsel was captured by a notorious knight and imprisoned in a castle surrounded by square moat. The moat was infested with extraordinarily hungry alligators for whom the prospect of a luncheon damsel brought enormous smiles to their green faces. The moat was 20 feet across, and no drawbridge existed because the evil knight took it with him (giving his horse a major hernia).

After a time, a good knight and his squire rode up and said, ``Hail sweet damsel, for I am here; and thou art there; now what are we going to do?''

The knight, though good, was not too bright and the consequently paced back and forth along the moat looking anxiously at the alligators and trying feebly to think of a plan. Then, on the shore the knight found two sturdy beams of wood suitable for walking across but lacking sufficient length. Alas, the moat was 20 feet across, and the beams were each only 19 feet and 8 inches wide. He tried to stretch them and tried to think. Neither effort proved successful. He had no nails, screws, saws, superglue, or any other method of joining the two beams to extend their length.

What to do? What to do? Fortunately, the damsel, after a suitable time to allow the good knight to attempt to solve the puzzle herself, was able to give the knight a few hints that enabled him to rescue her and carried her home to her own castle. How did the maiden advise the knight to accomplish the rescue?

\ifnudges
\hrulefill

Thinking about variations on a situation helps us understand which features are essential and which are unnecessary.  In this case we might consider a variation in which the damsel in distress is on the other side of a 20-foot river rather than surrounded by a square moat.  Unfortunately for the maiden, if she were separated from freedom by a river, she would be stuck because the two 19 foot beams, in the absence of tools, would not enable the knight to rescue her.  Somehow the square shape of the moat must come into play in the solution.

Looking at extremes is a potent technique of analysis in many situations and may be helpful here.  The extremes, ether geometrical ones as in this situation or conceptual extremes in other situations, frequently reveal features that are otherwise hidden.
\fi

\clearpage
\subsection{That's a Meanie Genie}

On an archaeological dig in the Highlands of Tibet, Alley discovered an ancient oil lamp. Just for laughs she rubbed the lamp. She quickly stopped laughing when the huge puff of a magenta smoke sprouted from the lamp, and an ornery genie in Murray appeared. Murray, looking at the stunned Alley, exclaimed, ``Well, what are you staring at? Okay, okay, you've found me; you get your three wishes. So what will they be?'' Alley, although in shock, realized what an incredible opportunity she had. Thinking quickly, she said, ``I'd like to find the Rama Nujan, the jewel that was first discovered by Hardy, the High Lama.'' ``You got it,'' replied Murray, and instantly nine identical looking stones appeared. Alley looked at the stones and was unable to differentiate anyone from the others. 

Finally, she said to Murray, ``So where is the Rama Nujan?'' Murray explained, ``It is embedded in one of the stones. You said you wished to find it. So now you have to find it. Oh, by the way, you may take only one of the stones with you, so you had best be careful how you choose!'' ``But they look identical to me. How will I know which one has the Rama Nujan in it?'' Alley questioned. ``Well, eight of the stones weigh the same, but the stone containing the jewel weighs slightly more than the others,'' Murray responded with a devilish grin.

Alley, now getting annoyed, whispered under her breath, ``Gee, I wish I had a balance scale.'' Suddenly a balance scale appeared. ``That was wish two!'' declared Murray. ``Hey, that's not fair!'' Alley cried. ``You want to talk fair? You think it's fair to be locked in a lamp for 1729 years? You know you can't get internet in there, and there's no room for a satellite dish! So don't talk to me about fair,'' Murray explained. Realizing he had gone a bit overboard, Murray proclaimed, ``Hey, I want to help you out, so let me give you a tip: that balance scale may only be used once.'' ``What? Only once?'' she said, thinking out loud. ``I wish I had another balance scale.'' ZAP! Another scale appeared. ``Okay, kiddo, that was wish three.'' Murray snickered. ``Hey, just one minute,'' Alley said now regretting not having asked for \$1 million or something more standard. ``Well at least this new scale works correctly, right?'' ``Sure, just like the other one. You may use it only once.'' ``Why?'' Alley inquired. ``Because it is a 'wished' balance scale. That means that you can only use it once since it was only one wish. It's just like you cannot wish for 100 more wishes.'' ``You are a very obnoxious genie.'' ``Hey, I don't make up the rules, lady, I just follow them.''

So, Alley may use each of the two balance scales exactly once. Is it possible for Alley to select the slightly heavier stone containing the Rama Nujan stone from among the nine identical looking stones? Please explain why or why not.

\ifnudges
\hrulefill

Initially, we might think that it is impossible to find the jewel since Alley is allowed to make only two comparisons.  Instead of comparing stones individually, perhaps she should compare one collection of stones with another collection of stones.  Now, suppose Alley compares one group with another using the first scale.  What can she conclude?  What should she do next?
\fi
\clearpage
\subsection{The Fountain of Knowledge}

During an incredibly elaborate hazing stunt during pledge week, Trey Sheik suddenly found himself alone in the Sahara desert. His desire to become a fraternity brother was now overshadowed by his desire to find something to drink (these desires, of course, are not unrelated). As he wandered aimlessly through the desert sands, he began to regret his involvement in the whole frat scene. Both hours and miles had passed and Trey was near dehydration. Only now did Trey appreciate the advantages of sobriety. Suddenly, as though it were a mirage, Trey came upon an oasis.

There, sitting in a shaded kiosk beside a small pool of mango nectar, was an old man named Al Donte. Big Al, not only ran the mango bar but was also a travel agent and could book Trey on a two-humped camel back to Michigan. At the moment, however, Trey desired nothing but a large drink of that beautifully translucent and refreshing mangoade. Al informed Trey that the juice was sold only in 8 ounce servings and that the cost for one serving was \$3.50. Trey frantically searched on his pockets and found some change and much sand. Trey counted and discovered that he had exactly \$3.50.

Trey's jubilation at the thought of liquid coating his dried and chapped throat was quickly shattered when Al casually announced that there were no 8- ounce glasses available. Al had only a 6- ounce glass and a 10- ounce glass -- neither of which would have any markings on them. Al, being a man of his word, would not hear of selling any more or any less than an 8- ounce serving of his libation. Trey, in desperation, wondered whether it was possible to use two glasses to produce exactly 8- ounces of mango juice in the 10- ounce glass. Trey thought and thought. Do you think it is possible to use only the unmarked 6- and 10- ounce glasses to produce exactly 8 ounces in the 10 ounce glass? If so, explain how, if not, explain why not.

\ifnudges
\hrulefill

Attempt this puzzle by trial and error together with careful observation.  As we observe the outcomes of various attempts we will teach ourselves what is possible.  Try filling up the 10-ounce glass, and then use it to fill the 6-ounce glass.  What do you now have -- anything new?
\fi

\clearpage
\subsection{Dodge Ball}

%Dodge Ball is a game for two players -- Player One and Player Two (although any two people can play it, even if they are not named ``Player One'' and ``Player Two'').  Each player has his or her own special board and given six turns.
%
%Player One begins by filling in the first horizontal row of their table with a run of X's and O's.  That is, on the first line of his board, he will write six letters -- one in each box -- each letter being either an X or an O.  Then Player Two places either one X or one O in the first box of their board.  So at this point, Player One has filled in the first row of their board with six letters, and Player Two has filled in the first box of their board with one letter.
%
%The game continues with Player One writing down a run of six letters (X's and O's), one in each box of the second horizontal row of their board, followed by Player Two writing one letter (an X or an O) in the second box of their board.  This game proceeds in this fashion until all Player One's boxes are filled with X's and O's; thus, Player One has produced six rows of six marks each, and Player Two has produced one row of six marks.  All marks are visible to both players at all times.  Player One wins if any horizontal row he wrote down is identical to the row that Player Two created (Player One matches Player Two).  Player Two wins if Player Two's string is not one of the six strings made by Player One (Player Two dodges Player One).
%
%Would you rather be Player One or Player Two?  Who has the advantage? Can you devise a strategy for either side that will always result in victory?

Directions for the game:  

Player One begins by filling in the first horizontal row of her table with X's and O's.  Then Player Two places either one X or one O in the first box of his board.  The game continues with Player One writing down a run of six X's and/or O's in the second horizontal row of her board, followed by Player Two writing an X or O in the second box of his board.  The game continues until all boxes on both boards are filled.

	Player One wins if any horizontal row she wrote down is identical to the row that Player Two created (Player One matches Player Two).  Player Two wins if his row is not one of the six rows made by Player One (Player Two dodges Player One).

\noindent Player One's board:
\begin{center}
\renewcommand{\arraystretch}{2}
	\begin{tabular}{|c|*{6}{p{.12\textwidth}|}}
	\hline
1&&&&&&\\	
	\hline
2&&&&&&\\	
	\hline
3&&&&&&\\	
	\hline
4&&&&&&\\	
	\hline
5&&&&&&\\	
	\hline
6&&&&&&\\	 \hline
	\end{tabular}
\end{center}

\noindent Player Two's board:
\begin{center}
\newcolumntype{C}[1]{>{\centering\arraybackslash}p{#1}}
\renewcommand{\arraystretch}{2}
	\begin{tabular}{|*{6}{C{.12\textwidth}|}}
	\hline 1&2&3&4&5&6\\
	\hline &&&&&\\	 \hline
	\end{tabular}
\end{center}

Would you rather be Player One or Player Two?  Who has the advantage?  Why?  Can you devise a strategy for either side that will always result in victory?  This little game holds within it the key to understanding the sizes of infinity. 
\clearpage

\noindent Player One's board:
\begin{center}
\renewcommand{\arraystretch}{2}
	\begin{tabular}{|c|*{6}{p{.12\textwidth}|}}
	\hline
1&&&&&&\\	
	\hline
2&&&&&&\\	
	\hline
3&&&&&&\\	
	\hline
4&&&&&&\\	
	\hline
5&&&&&&\\	
	\hline
6&&&&&&\\	 \hline
	\end{tabular}
\end{center}

\noindent Player Two's board:
\begin{center}
\newcolumntype{C}[1]{>{\centering\arraybackslash}p{#1}}
\renewcommand{\arraystretch}{2}
	\begin{tabular}{|*{6}{C{.12\textwidth}|}}
	\hline 1&2&3&4&5&6\\
	\hline &&&&&\\	 \hline
	\end{tabular}
\end{center}

\vfill

\noindent Player One's board:
\begin{center}
\renewcommand{\arraystretch}{2}
	\begin{tabular}{|c|*{6}{p{.12\textwidth}|}}
	\hline
1&&&&&&\\	
	\hline
2&&&&&&\\	
	\hline
3&&&&&&\\	
	\hline
4&&&&&&\\	
	\hline
5&&&&&&\\	
	\hline
6&&&&&&\\	 \hline
	\end{tabular}
\end{center}

\noindent Player Two's board:
\begin{center}
\newcolumntype{C}[1]{>{\centering\arraybackslash}p{#1}}
\renewcommand{\arraystretch}{2}
	\begin{tabular}{|*{6}{C{.12\textwidth}|}}
	\hline 1&2&3&4&5&6\\
	\hline &&&&&\\	 \hline
	\end{tabular}
\end{center}
\clearpage
\subsection{Dot of Fortune}

One day three college students were selected at random from the studio audience to play the ever-popular TV game show, ``Dot of Fortune.'' One of the students already had discovered the power and beauty of mathematical thinking, while the other two were not nearly so fortunate.  The stage had no mirrors, reflecting surfaces, or television monitors.  The three students were led blindfolded to their places around a small round table.  As the rules of the game were explained by Pat, Vanna affixed to each of the three youthful foreheads a conspicuous but small colored dot.

``So, contestants,'' Pat explained, ``at the bell your blindfolds will be removed.  You will see your two companions sitting quietly at the table, each with a dot on his or her forehead.  Each dot is either red or white.  You cannot, of course, see the dot on your own forehead.  After you have observed the dots on your companions' foreheads, you will raise your hand if you see at least one red dot. If you do not see a red dot, you will keep your hands on the table. The object of the game is to deduce the color of your own dot.  As soon as you know the color of your dot, you are to hit the buzzer in front of you.  Do you understand the rules of the game?'' All the students understood the rules, although the math fan understood them better.

``Are you ready?'' asked Vanna after affixing three red dots to the foreheads of the three students.  After the three contestants nodded, Vanna instructed them to simultaneously remove their blindfolds as the studio audience quivered with anticipation.  The three students looked at one another's dots, and all raised their hands.  After some time, the math fan hit her buzzer knowing what color dot she had.  Please explain how she knew this.  Why did the other students not know?

\ifnudges
\hrulefill

Sometimes no action is action enough.  Put yourself in the position of one of the three contestants.  You know that the dot on your forehead is either red or white.  The trick to figuring out this conundrum is to suppose you have a white dot and see what would happen.

Suppose you are sitting at the table looking at two red dots, and you assume that you have a white dot on your forehead.  What would each of the two others at the table see?  What could they deduce?  What would they do?  What did they do? What can you conclude?
\fi
