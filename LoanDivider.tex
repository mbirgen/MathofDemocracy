\subsection{Lone Divider} \label{sec:LoneDivider}
%\clearpage
\item Three players (Anna, Ben, and Cara) must divide cake among themselves. Suppose the cake is divided into three slices ($s_1$, $s_2$, and $s_3$). The values of the entire cake and at each of the three slices in the eyes of each of the players are shown in the following table.

\begin{center}
	\begin{tabular}{ccccc}
\hline
& Whole cake & $s_1$ & $s_2$ &$s_3$\\\hline
Anna & \$12.00 & \$3.00 & \$5.00 & \$4.00 \\\hline
Ben & \$15.00 & \$4.00 & \$4.50 & \$6.50 \\\hline
Cara & \$13.50 & \$4.50 & \$4.50 & \$4.50 \\\hline 
\end{tabular}

\end{center}
 \begin{enumerate}
	\item Indicate which of the three slices are fair shares to Anna. \fillwithlines{\stretch{1}}
	\item Indicate which of the three slices are fair shares to Ben.\fillwithlines{\stretch{1}}
	\item Indicate which of the three slices are fair shares Cara.\fillwithlines{\stretch{1}}
	\item How do you know which player divided the cake into three equal pieces?\fillwithlines{\stretch{1}}
	\item Describe a fair division of the cake.\fillwithlines{\stretch{1}}
\end{enumerate}
\item \boxedblank[2in]{\textbf{Lone Divider:}\ifsolns One of the players is assigned to be the divider.  The player divides the booty into equal shares (according to him).  The other players are called the choosers.  They indicate which of the shares are fair according to them.  The booty is split by giving each of the choosers one of the shares they have indicated and the divider gets the remaining share.\else \fillwithlines{\stretch{1}}\fi} 
%Make this two sections where one section is lone dider easy and the other is what to do when more than one player only bids on one plot.
\vfill \index{lone divider}

\clearpage

\item Three partners (David, Carol, and Charlotte) are dividing the plot of land among themselves using the loan divider method. Using a map, the divider (David) divides the property into three parcels $s_1$, $s_2$, and $s_3$. When the choosers bid lists are open, Carol's bid list is $\{s_1,s_2,s_3\}$ and Charlotte's bid list is $\{s_1\}$.
\begin{enumerate}
	\item Describe the fair division where David's fair share is $s_2$ \fillwithlines{\stretch{1}}
	\item Describe the fair division where David's fair share is $s_3$\fillwithlines{\stretch{1}}
\end{enumerate}


\item What should you do if both Carol and Charlotte bid $\{s_1,s_2\}$?
\fillwithlines{\stretch{1}}


\item How would you run a fair division if there were four people trying to divide an item? \fillwithlines{\stretch{2}}

\clearpage
\item We have one divider, Demon and three choosers, Chuck, Charlie, and Charles.  Demon divides the cake into four shares, $s_1, s_2, s_3$ and $s_4$.  The following table shows how each of the players values each of the four shares.  Remember that this information is private and not known to the other players.

\begin{center}
	\begin{tabular}{lcccc}
\hline
& $s_1$&$s_2$&$s_3$ & $s_4$\\\hline
Demon & 25\% & 25\% & 25\% & 25\% \\\hline
Chuck & 30\% & 20\% & 35\% & 15\% \\\hline
Charlie & 20\% & 20\% & 40\% & 20\% \\\hline
Charles & 25\% & 20\% & 20\% & 35\% \\\hline
\end{tabular}
\end{center}
\begin{enumerate}
	\item What are each player's bidding lists?
	\begin{center}
	\begin{tabular}{lp{2in}}
\hline
\ifsolns
	Demon & $s_1,s_2, s_3, s_4$\\\hline
	Chuck & $s_1, s_3$\\\hline
	Charlie &  $s_3$\\\hline
	Charles &  $s_1, s_4$\\\hline
\else
	Demon & \hspace{2in}\\\hline
	Chuck & \\\hline
	Charlie &\\\hline
	Charles &\\\hline
\fi
\end{tabular}
\end{center}
	
	\item What is the distribution
	
	Demon  \par
Chuck  \par
Charlie   \par
Charles   \par
\end{enumerate}
\fillwithlines{\stretch{1}}

\end{enumerate}

\clearpage
\HOMEWORK
%</HWHEADER>

%<*HOMEWORK>


\begin{Denumerate}
  \item Three partners (David, Carol, and Charlotte) are dividing the plot of land among themselves using the loan divider method. Using a map, the divider (David) divides the property into three parcels $s_1$, $s_2$, and $s_3$. When the choosers bid lists are open, Carol's bid list $\{s_2,s_3\}$ and Charlotte's bid list is $\{s_1, s_3\}$.
		\begin{enumerate}
			\item Describe the fair division where David's fair share is $s_1$.
			\solution*{ \fbox{David - $s_1$, Carol - $s_2$, Charlotte - $s_3$} Because David has a fair share of $s_1$, the only other option for Charlotte is $s_3$. Since Carol will take $s_2$ or $s_3$, and $s_3$ is taken, Carol gets $s_2$.
                }\vfill
			\item Describe the fair division where David's fair share is $s_2$.
						\solution*{ \fbox{David - $s_2$, Carol - $s_3$, Charlotte - $s_1$} Because David has a fair share of $s_1$, the only other option for Charlotte is $s_3$. Since Carol will take $s_2$ or $s_3$, and $s_3$ is taken, Carol gets $s_2$.
                }\vfill
			\item Describe the fair division where David's fair share is $s_3$.
			\solution{\fbox{David - $s_3$, Carol - $s_2$, Charlotte - $s_1$}}\vfill
		\end{enumerate}
%  \begin{enumerate}
%          \item 
%                
%          \item 
%          			\solution*{
%                  \fbox{} 
%                }
%        \end{enumerate}

\hwnewpage
\item Four partners (Evan, Fiona, Greg, and Harry) are dividing a piece of land valued at \$480,000 among themselves using the lone-divider method. Using a map, the divider divides the property into four parcels, $s_1,s_2,s_3$ and $s_4$. The following table shows the value of each of the four parcels in the eyes of each partner, but some of the information in the table is missing.

\begin{center}
	\begin{tabular}{lcccc}
&$s_1$&$s_2 $&$ s_3$ & $s_4$\\\hline\ifsolns
Evan  & \$80,000 & \$85,000 & \textbf{\$120,000} & \$195,000\\\hline
Fiona  & \textbf{\$125,000} & \$100,000 & \$135,000 & \$120,000\\\hline
Greg  & \$120,000 & \textbf{\$120,000} & \$120,000 & \textbf{\$120,000}\\\hline
Harry  & \$95,000 & \$100,000 & \textbf{\$175,000} & \$110,000 \\\hline
\else
Evan & \$80,000 & \$85,000 & & \$195,000\\\hline
Fiona & & \$100,000 & \$135,000 & \$120,000 \\\hline
Greg & \$120,000 & & \$120,000 &\\\hline
Harry & \$95,000 & \$100,000 & & \$110,000 \\\hline\fi
\end{tabular}

\end{center}
\begin{enumerate}
	\item Who was the divider? Explain. \solution{Greg because he has the same amount for every piece.}\vfill
	\item Describe the choosers' respective bid lists.
	\ifsolns
	\par
	\begin{tabular}{lcccc}
	Evan  &  &  & $s_3$ & $s_4$\\
Fiona  & $s_1$ &  & $s_3$ & $s_4$\\
Greg  & $s_1$ & $s_2$ & $s_3$ & $s_4$\\
Harry  &  &  & $s_3$ & \\
	\end{tabular}\fi\vfill
	\item Described a fair division of property. \solution{\par Evan     $s_4$\\Fiona  $s_1$   \\Greg   $s_2$  \\Harry    $s_3$}
	\item Explain why your answer above is the only possible fair division of the property.
	\solution{\fbox{ Basically, because Harry only bids on one property, \par he starts the ball rolling.}}\vfill
\end{enumerate}
\end{Denumerate} \ENDHOMEWORK

\clearpage
%%%%%%%%%%%%%%%%%%%%%%%%%%%%%%%%%%%%%%%%%%%%%%%%%%%%%%%%%%%%%%%%%%%%%%%%%%%%%%%%%%%%%%%%%%%%%%%%%%%%%%
\subsection{Lone-Divider Case 2}

There is one complication that can happen with the lone-divider method we have not yet discussed.  It happens when two players want one and only one of the shares.  This can be solved, but requires a little more analysis.

\begin{enumerate}
	\item Dale divides the cake into three pieces $s_1$, $s_2$, and $s_3$.  The table below shows the value of the three pieces in the eyes of each of the players.
	
	\begin{center}
	\begin{tabular}{lccc}
	\hline
	 & $s_1$ & $s_2$ & $s_3$ \\\hline
	Dale & $33\frac13$\% & $33\frac13$\% & $33\frac13$\% \\\hline
	Cindy & 20\% & 30 \% & 50\% \\\hline
	Cher & 10\% & 20\% & 70\% \\\hline
\end{tabular}
\end{center}

\begin{enumerate}
	\item Describe the choosers' respective bid lists.
	\ifsolns
		Dale  $\qquad s_1,s_2, s_3$\\
	Cindy $\qquad  s_3$\\
	Cher $\qquad  s_3$\\
	\else
	\fillwithlines{\stretch{1}}
	\fi
		\item What problem do you run into when you try to divide the cake? \ifsolns \par Both Cindy and Cher only want the last piece. \else \fillwithlines{\stretch{1}}\fi
		%\vfill
%	\item Described a fair division of property.
What we are going to do is to give Dale one of the pieces that Cindy and Cher do not want, say slice $s_2$.
	\item How much are $s_1$ and $s_3$ worth together to Cindy?  to Cher? \ifsolns \par 70\% and 80\% \fi
	\fillwithlines{\stretch{1}}
	\item What you are going to to is combine $s_1$ and $s_3$ and give them to Cindy and Cher to perform a Divider-Chooser.  Explain why this will result in both Cindy and Cher receiving at least $33\frac13$\% of the cake in their eyes.  \fillwithlines{\stretch{1}}
	\item In fact, what is the minimum percentage Cindy will receive?  What is the minimum percentage Cher will receive? \ifsolns \par 35\% and 40\%\else \fillwithlines{\stretch{1}} \fi %\vfill

\end{enumerate}
%\vfill
\clearpage

	\item Three partners (David, Carol, and Charlotte) are dividing the plot of land among themselves using the loan divider method. Using a map, the divider (David) divides the property into three parcels $s_1$, $s_2$, and $s_3$. When the choosers bid lists are open, Carol's bid list $\{s_1\}$ and Charlotte's bid list is $\{s_1\}$.
\begin{enumerate}
	\item Considering that Carol only bid on the first parcel, explain how we know that parcel 2 is worth less than $1/3$ of the value of the land to Carol. Explain how we know that parcel 3 is worth less than $1/3$ of the value of the land to Carol.  \ifsolns \par \fbox{Because if they were worth at least $1/3$ of the value, they would be on the bid list.}\else \fillwithlines{\stretch{1}}\fi 
	\item Using the above information, explain how the combination of the first two parcels is worth more than $2/3$ of the value of the land to Carol. \ifsolns \par \fbox{Because what is left over is worth less than $1/3$.}\else \fillwithlines{\stretch{1}}\fi
	\item Using similar logic, explain how the combination of the first two parcels is worth more than $2/3$ of the value of the land to Charlotte.\fillwithlines{\stretch{1}}
\end{enumerate}


\end{enumerate}

\clearpage
%%%%%%%%%%%%%%%%%%%%%%%%%%%%%%%%%%%%%%%%%%%%%%%%%%%%%%%%%%%%%%%%%%%%%%%%%%%%%%%%%%%%%%%%%%%%%%%%%%%%%%
\HOMEWORK
\begin{Denumerate}

\item Six players (D,$C_1,C_2,C_3,C_4$ and $C_5$) are dividing a cake among themselves using the lone-divider method. D cuts the cake into six slices $s_1,s_2,s_3,s_4,s_5,$ and $s_6$. When the choosers' bid lists are opened, $C_1$'s bid list is $\{s_1\}$, $C_2$'s bid list is $\{s_2,s_3\}$, $C_3$'s bid list is $\{s_4,s_5\}$, $C_4$'s bid list is $\{s_4,s_5\}$, and $C_5$'s bid list is $\{s_1\}$. Describe how to proceed to obtain a fair division of the cake.
\solution{ There are  many correct solutions.  They all include $C_2$ getting either $s_2$ or $s_3$.  
$C_3$ and $C_4$ getting either $s_4$ or $s_5$ and the other one going to the other, and $C_1$ and $C_5$
splitting the combination of $s_1$ and another piece which are then split using divider-chooser.}
\vfill

\item We have one divider, Demon and three choosers, Chuck, Charlie, and Charles.  Demon divides the cake into four shares, $s_1, s_2, s_3$ and $s_4$.  The following table shows how each of the players values each of the four shares.  Remember that this information is private and not known to the other players.

\begin{center}
	\begin{tabular}{lcccc}
\hline
& $s_1$&$s_2$&$s_3$ & $s_4$\\\hline
Demon & 25\% & 25\% & 25\% & 25\% \\\hline
Chuck & 15\% & 20\% & 50\% & 15\% \\\hline
Charlie & 20\% & 20\% & 40\% & 20\% \\\hline
Charles & 25\% & 20\% & 20\% & 35\% \\\hline
\end{tabular}
\end{center}
\begin{enumerate}
	\item What are each player's bidding lists?
	
	\begin{center}
	\ifsolns
	\begin{tabular}{lcccc}
	Demon  &  $s_1$ & $s_2$ & $s_3$  &  $s_4$\\
Chuck  &  &  & $s_3$  & \\
Charlie  &  &  & $s_3$  & \\
Charles  &  $s_1$ &  &  &  $s_4$ \\
	\end{tabular}
	\else
	\begin{tabular}{lp{2in}}
\hline
%& $s_1$&$s_2$&$s_3$ & $s_4$\\\hline
Demon & \\\hline
Chuck & \\\hline
Charlie &  \\\hline
Charles &  \\\hline
\end{tabular}\fi
\end{center}
	
	\item How are you going to make a fair division?
	\solution{Similar to the previous problem, Charles gets	 $s_1$ or	 $s_4$, Demon gets something, and Chuck and Charlie will take the combination of $s_3$ and another slice and perform divider-chooser.}
	\vfill
\end{enumerate}




\end{Denumerate} \ENDHOMEWORK
%%%%%%%%%%%%%%%%%%%%%%%%%%%%%%%%%%%%%%%%%%%%%%%%%%%%%%%%%%%%%%%%%%%%%%%%%%%%%%%%%%%%%%%%%%%%%%%%%%
%\clearpage
%\subsection{Lone Chooser}
%
%The lone chooser method was proposed in 1964 by A.M. Fink, a mathematician at Iowa State University and a Wartburg College graduate.  The lone chooser method is an iterative process that generalizes to multiple players more easily than the lone divider method. 
%
%\begin{enumerate}
%	\item \vfill
%	\item \vfill
%	\item \vfill
%
%\end{enumerate}
%
%\clearpage

\clearpage
