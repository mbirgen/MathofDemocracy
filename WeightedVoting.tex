\section{Weighted Voting Systems} \label{sec:WeightedVoting}

\begin{enumerate}
  \item Notes on weighted voting systems: \index{weighted voting}
	
	\begin{itemize}
		\item $P_i$ \vfill
		\item $w_i$ \vfill
		\item $q$\vfill
		\item $[q|w_1,w_2,\dots,w_n]$ \vfill
	\end{itemize}

\ifsolns \par\soln
	Weighted voting systems are when there are $n$ voters labeled $P_1\dots P_n$, and each voter casts a vote with a certain weight, $w_i$.  Think of this as having each voter being able to cast $w_i$ votes all for the same decision.  In order for a motion to pass, the sum of all the votes needs to be at least $q$, the quota.  We need the quota to be more than 50\% of the sum of all the votes and no more than the sum of all the votes, but it can be anything in between.  We give a weighted voting system as 
	$[q|w_1,w_2,\dots,w_n]$.
\else          \vfill \fillwithlines{\stretch{1}}\fi
  \clearpage
  \item In a weighted voting system with weights $[30, 29, 16, 8, 3, 1]$,
        if a two-thirds majority of votes is needed to pass a motion, what is the quota? \index{quota}
        \ifsolns
					\fbox{58}
				\else
					          \fillwithlines{\stretch{1}}
				\fi

  \item Consider the weighted voting system $[14, 9, 8 ,5]$.
		\begin{enumerate}
          \item What is the largest reasonable quota for this system?
                  \ifsolns
					\fbox{36}
				\else
					          \fillwithlines{\stretch{1}}
				\fi
          \item What is the smallest reasonable quota for this system?
                  \ifsolns
					\fbox{19}
				\else
					          \fillwithlines{\stretch{1}}
				\fi
		\end{enumerate}
  \item Consider the weighted voting system $[20|7,5,4,4,2,2,2,1,1]$.
        \begin{enumerate}
          \item How many voters are there?
        \ifsolns
					\fbox{9}
				\else
					          \fillwithlines{\stretch{1}}
				\fi
          \item What is the quota?
        \ifsolns
					\fbox{20}
				\else
					          \fillwithlines{\stretch{1}}
				\fi
          \item What is the weight for voter $P_2$?
         \ifsolns
					\fbox{5}
				\else
					          \fillwithlines{\stretch{1}}
				\fi
         \item If the first 4 voters vote for a motion and the rest vote against, does the motion pass?
                \ifsolns
					\fbox{Yes}
				\else
					          \fillwithlines{\stretch{1}}
				\fi
        
          \item If $P_1$ and $P_2$ vote against a motion, will the motion pass?
        \ifsolns
					\fbox{No}
				\else
					          \fillwithlines{\stretch{1}}
				\fi
        \end{enumerate}

\clearpage

  \item What is peculiar about each of the following weighted voting systems?
	
        \begin{enumerate}
        \item $[20|10, 10, 9]$
					\ifsolns
						\fbox{Cannot win without $P_1$ and $P_2$.}
					\else
						\vfill          \fillwithlines{\stretch{1}}
					\fi
				%\end{enumerate}
		
        \item $[7|4, 2, 1]$
        \ifsolns
					\fbox{Everyone must vote for the motion to pass}
				\else
					\vfill          \fillwithlines{\stretch{1}}
				\fi
        %\end{enumerate}

		\item $[51|50, 49, 1]$
        \ifsolns
					\fbox{First voter must vote for the motion for the motion to pass.}
				\else
					\vfill          \fillwithlines{\stretch{1}}
				\fi
        %\end{enumerate}
		
        \item $[6|6, 2, 1, 1]$
        \ifsolns
					\fbox{First voter is the only one who makes a difference.}
				\else
					\vfill          \fillwithlines{\stretch{1}}
				\fi
        %\end{enumerate}
		
		\end{enumerate}
		
\pagebreak
    \item A \defnstyle{dummy} is \ldots 
		    \ifsolns
					\par\soln A voter whose vote does not make any difference on whether a motion passes or not.
				\fi
        %\end{enumerate}
          \fillwithlines{\stretch{1}} \index{dummy}
	
    \item A \defnstyle{dictator} is \ldots 
		    \ifsolns
					\par\soln A voter where their vote is the only reason a motion passes.
				\fi
		          \fillwithlines{\stretch{1}} \index{dictator}
    \item A voter has \defnstyle{veto power} if \ldots 
		    \ifsolns
					\par\soln A voter whose must vote for a motion in order for it to pass.
				\fi
		          \fillwithlines{\stretch{1}} \index{veto power}
    \item In the weighted voting system $[12|9,5,4,2]$, are there any dummies or dictators?
                    \vspace{1.5in}

    \item In designing a weighted voting system $[q|6,5,4,3,2,1]$, what is the largest quota $q$
          you could pick without giving veto power to anyone?
                    \vspace{1.5in}
    
    \item In the weighted voting system $[q|8,5,4,1]$, if every voter has veto power, what is the quota $q$?
                    \vspace{1.5in}

\clearpage
\item A committee has four members ($P_1, P_2, P_3,$ and $P_4$).  In this committee, $P_1$ has twice as many votes as $P_2$; $P_2$ has twice as many votes as $P_3$: $P_3$ has twice as many votes as $P_4$.  Describe the committee as a weighted voting system when the requirements to pass a motion are
\begin{enumerate}
	\item at least two-thirds of the votes 
		\ifsolns \fbox{[10|8,4,2,1]} \else \vfill           \fillwithlines{\stretch{1}} \fi
	\item more than two-thirds of the votes 
		\ifsolns \fbox{[11|8,4,2,1]} \else \vfill           \fillwithlines{\stretch{1}}\fi
	\item at least 80\% of the votes 
		\ifsolns \fbox{[12|8,4,2,1]} \else \vfill           \fillwithlines{\stretch{1}}\fi
	\item more than 80\% of the votes
		\ifsolns \fbox{[13|8,4,2,1]} \else \vfill           \fillwithlines{\stretch{1}}\fi
\end{enumerate}
\end{enumerate}

%</WORKSHEETS>

%<*HWHEADER>
\HOMEWORK
%</HWHEADER>

%<*HOMEWORK>

\begin{Venumerate}

  \item Alice, Bob, Charles, and Danielle are the stockholders in Alphabet Industries, Inc.
        Alice owns 252 shares, Bob owns 741 shares, Charles inherited 637 shares,
        and 412 shares are in Danielle's hands.
        As usual, each share corresponds to a vote in the stockholder's meeting.
        \begin{enumerate}
          \item If a certain type of motion requires a majority vote,
                what is the smallest number of votes needed to pass the motion?
                \solution*{%
                  \fbox{1022.}  
                  \ifgradersolns
                    (Partial credit for 1021.)
                  \fi
                }\vfill           \fillwithlines{\stretch{1}}
                
          \item A different type of motion requires a $2/3$ vote to pass.
                What is the smallest number of votes needed to pass this motion?
                \solution{\fbox{1362.}} \vfill           \fillwithlines{\stretch{1}}
          \item Using the quota you found in part (b), express the weighted voting system
                in the correct notation (with brackets and quota).
                \solution*{\fbox{$[1362 | 741, 637, 412, 252]$.}}\vfill           \fillwithlines{\stretch{1}}
        \end{enumerate}
  \item Which voters have veto power in the system $[51 | 29, 21, 8, 3, 1]$?
                \ifsolns
                  \par\soln \fbox{$P_1$ and $P_2$} (the 29 and the 21) have veto power.
                \fi             \vfill%\fillwithlines{\stretch{1}}


\hwnewpage
  \item Find all dictators, dummies, and voters with veto power in the following weighted voting systems:
        \begin{enumerate}
          \item $[51 | 20, 20, 20]$
                \solution{No dictators, no dummies, all three have veto power.}   \vfill       %\fillwithlines{\stretch{1}}
          \item $[51 | 36, 34, 23, 6]$
                \solution*{No dictators, $P_4$ (the 6) is a dummy, no one has veto power.}        \vfill     %\fillwithlines{\stretch{1}}
          \item $[25 | 27, 11, 7, 2]$
                \solution*{$P_1$ (the 27) is a dictator (and so has veto power); the rest are dummies.}   \vfill%          \fillwithlines{\stretch{1}}
          \item $[31 | 15, 13, 6, 4, 2]$
                \solution{No dictators;
                        $P_1$ and $P_2$ (the 15 and the 13) have veto power;
                        $P_5$ (the 2) is a dummy.}            \vfill% \fillwithlines{\stretch{1}}
        \end{enumerate}

%\hwnewpage
	\item In 1958, the Treaty of Rome established the European Economic Community (EEC) and instituted a
          weighted voting system for the EEC's governance.  
          The members at that time were France, Germany, Italy, Belgium, the Netherlands, and Luxembourg.  
          The three largest countries (France, Germany and Italy) were each given a vote with
          weight 4, Belgium and the Netherlands had votes of weight 2 and Luxembourg's vote had weight 1.  
          The quota was 12.
		
          What is unusual or interesting about this weighted voting system?
          \ifsolns
            \par\soln \fbox{Luxembourg is a dummy.}
          \fi
	      \vfill           \fillwithlines{\stretch{2}}

\end{Venumerate}

\ENDHOMEWORK %</HOMEWORK>

%<*WORKSHEETS>

\cleartooddpage