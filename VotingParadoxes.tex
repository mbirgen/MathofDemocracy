\section{Voting Paradoxes and Problems} \label{sec:VotingParadoxes}

\begin{enumerate}
  \item Of the four voting systems we've studied, which is best?  Why?
        \fillwithlines{\stretch{1}}

  %PLURALITY METHOD FAILS HEAD-TO-HEAD CRITERION
  \item The Des Moines Running Club is electing a new team captain.
        There are three candidates:  Irving, Joseph, and Karl.
        The preference table is the following.
        \begin{center}
          \begin{tabular}{|c|c|c|c|c|c|c|c} \hline
                       & \multicolumn{3}{c|}{\# of Ballots} \\
            Ranking    & 4 & 3 & 2 \\ \hline\hline
            1st choice & J & I & K  \\
            2nd choice & I & J & I  \\
            3rd choice & K & K & J  \\ \hline
          \end{tabular}
        \end{center}
        \begin{enumerate}
          \item If the club uses the plurality method,
                who will win the election? \ifsolns\fbox J \fi
                \fillwithlines{\stretch{1}}
          \item If it were a head-to-head race between Irving and Joseph, who would win? \ifsolns\fbox I \fi
                \fillwithlines{\stretch{1}}
          \item If it were a head-to-head race between Irving and Karl, who would win? \ifsolns\fbox I \fi
                \fillwithlines{\stretch{1}}
          \item If it were a head-to-head race between Joseph and Karl, who would win? \ifsolns\fbox J \fi
                \fillwithlines{\stretch{1}}
          \item Why does the plurality method's result seem unfair in this election?
                \ifsolns Irving defeats Joseph and Karl in head-to-head matches, but Joseph wins via plurality.\fi
                \fillwithlines{\stretch{1}}
        \end{enumerate}


\clearpage

  \item \boxedblank{\textbf{Head-to-Head Criterion:} \ifsolns \par\soln A candidate who beats all the other candidates head to head should win the election. \fi          \fillwithlines{\stretch{1}}} \index{criterion!head to head}

  %BORDA COUNT FAILS MAJORITY CRITERION
  \item The school board is voting whether to award the contract for building a new elementary school
        to Aardvark Architects,
        to Bob's Builders,
        to the Confident Construction Company,
        or to Davenfield Developers.
        \begin{center}
          \begin{tabular}{|c|c|c|c|c|c|c|c} \hline
                       & \multicolumn{4}{c|}{\# of Ballots} \\
            Ranking    & 5 & 3 & 3 & 1   \\ \hline\hline
            1st choice & B & B & C & A  \\
            2nd choice & C & C & A & D   \\
            3rd choice & D & A & D & C   \\
            4th choice & A & D & B & B   \\ \hline
          \end{tabular}
        \end{center}
      \begin{enumerate}
        \item The school board votes using a Borda count.  \\
              Which company will get the contract? %Company C
              \ifsolns
                \par\soln $\begin{array}{clcr}
                             A & 4+9+6+5 &=& 24 \\
                             B & 32+4    &=& 36 \\
                             C & 24+12+2 &=& 38 \\
                             D & 3+16+3  &=& 25 \\
                           \end{array}$ 
                \qquad so C wins
              \fi
              \fillwithlines{\stretch{2}}
        \item Does this seem fair?  Why or why not?
              \ifsolns
                \par\soln No; two-thirds of the voters picked B as their first choice!
              \else
                \fillwithlines{\stretch{1}}
              \fi
      \end{enumerate}  

\clearpage
  \item \boxedblank{\textbf{Majority Criterion:} \ifsolns A candidate that wins a majority of first place votes should win the election. \fi} \index{criterion!majority}
  %PLURALITY WITH ELIMINATION FAILS MONOTONICITY

\ifodd\value{page}\clearpage\fi

  \item The board of regents of
        Western Iowa State at Sema
        will be voting tomorrow for the University's new president
        from four candidates: 
           Zehnpfennig,     \def\A{Z}\def\AA{Zehnpfennig}%
           Gersbach,        \def\B{G}\def\BB{Gersbach}%
           Haider,          \def\C{H}\def\CC{Haider}%
           and Rentmeester. \def\D{R}\def\DD{Rentmeester}%
        As of this moment, the voters' preferences are as follows:
        \begin{center}
          \begin{tabular}{|c|c|c|c|c|c|c|c} \hline
                       & \multicolumn{4}{c|}{\# of Ballots} \\
            Ranking    & 7 & 5 & 4 & 1   \\ \hline\hline
            1st choice & \A & \C & \B & \D  \\
            2nd choice & \D & \A & \C & \B   \\
            3rd choice & \B & \B & \D & \A   \\
            4th choice & \C & \D & \A & \C   \\ \hline
          \end{tabular}
        \end{center}
        The WIS regents use the plurality with elimination method.
        \begin{enumerate}
          \item If the vote were held today, who would be chosen as university president? 
                \ifsolns
                  \par\soln
                  \[\begin{array}{cc}Z& 7 \\ R&1 \\ G&4 \\ H&5\end{array}
                    \leadsto
                    \begin{array}{cc}Z&7 \\ G&5 \\ H&5 \end{array}
                    \leadsto Z\]
                \else
								\vfill
                  \fillwithlines{\stretch{1}}
                \fi
        \clearpage
          \item After thinking it over in bed tonight, 
                the single voter in the last column decides 
                that \AA\ really is the best candidate after all, and so his preferences change to \A,\D,\B,\C.
                Thus when the election is held tomorrow, the preference table is
                        \newlength{\mywidth}
                        \def\no#1{\settowidth{\mywidth}{#1}%
                                  %\makebox[0pt][l]{\raisebox{1.0ex}{\rule{\mywidth}{0.5pt}}}%
                                  %\makebox[0pt][l]{\raisebox{0.8ex}{\rule{\mywidth}{0.5pt}}}%
                                  \makebox[0pt][l]{\makebox[\mywidth][c]{$\times$}}%
                                  %\makebox[0pt][l]{\raisebox{0.6ex}{\rule{\mywidth}{0.5pt}}}%
                                  #1}
                        \begin{center}
                          \begin{tabular}{|c|c|c|c|c|c|c|c} \hline
                                       & \multicolumn{4}{c|}{\# of Ballots} \\
                            Ranking    & 7 & 5 & 4 & 1   \\ \hline\hline
                            1st choice & \A & \C & \B & \no{\D}  \A \\
                            2nd choice & \D & \A & \C & \no{\B}  \D \\
                            3rd choice & \B & \B & \D & \no{\A}  \B \\
                            4th choice & \C & \D & \A & \no{\C}  \C \\ \hline
                          \end{tabular}
                        \end{center}
                Who will be elected president?
                \ifsolns
                  \par\soln
                  $$\begin{array}{cc}Z& 8 \\ R&0 \\ G&4 \\ H&5\end{array}
                    \leadsto
                    \begin{array}{cc}Z& 8 \\ G&4 \\ H&5\end{array}
                    \leadsto
                    \begin{array}{cc}Z&8 \\ H&9 \end{array}
                    \leadsto H$$
                \else
                  \fillwithlines{\stretch{2}}
                \fi
                
          \item What is odd about your answers to (a) and (b)?
                \fillwithlines{\stretch{1}}
        \end{enumerate}

\item \boxedblank{\textbf{Monotonicity Criterion:} \ifsolns If the only changes in voting are in favor of the winning candidate, that candidate should still win the election. \fi} \fillwithlines{\stretch{1}}\index{criterion!monotonicity}

\clearpage
%CONDORCET WINNER MAY NOT EXIST---IN FACT, NONTRANSITIVITY
  \item Voters are trying to decide what to do with a \$50 million surplus in the state budget.
        Their options are: 
        \begin{itemize}
          \item[(T)] Give the money back to taxpayers as property tax relief.
          \item[(R)] Spend it on road construction and other infrastructure.
          \item[(H)] Build a new state historical building.
        \end{itemize}
        Here is the preference table:
                        \begin{center}
                          \begin{tabular}{|c|c|c|c|c|c|c|c} \hline
                                       & \multicolumn{3}{c|}{Number of Ballots} \\
                            Ranking    &  70,000 & 50,000 & 40,000    \\ \hline\hline
                            1st choice & T & R & H \\
                            2nd choice & R & H & T \\
                            3rd choice & H & T & R \\ \hline
                          \end{tabular}
                        \end{center}
        \begin{enumerate}
          \item The election is run by pairwise comparison.  What is the result?
                \ifsolns
                  \par\soln $R>H>T>R$.
                \fi
                \vfill\fillwithlines{\stretch{2}}
          \item What is odd about your results from part (a)?
					\ifsolns Try having candidate $R$ drop out of the race. \fi\fillwithlines{\stretch{1}}
        \end{enumerate}
        
\clearpage

  \item \boxedblank{\textbf{Irrelevant Alternatives Criterion:} \ifsolns \par\soln If a non-winning candidate leaves the election, the winner should not change, \fi\fillwithlines{\stretch{1}}} \index{criterion!irrelevant alternatives}

  \item Approval Voting: \index{approval voting}


	\ifsolns Vote for all candidates you approve of. \fi
        \fillwithlines{\stretch{1}}

  \item \boxedblank[2in]{\textbf{Arrow's Impossibility Theorem:} \ifsolns There is no way of counting votes that does not violate at least one of the criteria except for a Dictator. \fi\fillwithlines{\stretch{1}}} \index{Arrow's Impossibility Theorem}
%%%%%%%%%%%%%%%%%%%%%%%%%%%%%%%%%%%%%%%%%%%%%%%%%%%%%%%%%%%%%%%%%%%%%%%%%%%%%%%%%%%%%%%%%%




\end{enumerate}


%</WORKSHEETS>

%<*HWHEADER>
\cleartooddpage
\HOMEWORK
%</HWHEADER>

%<*HOMEWORK>


\begin{Venumerate}
  \item Consider the following preference matrix:
        \def\A{A} \def\B{B} \def\C{C} \def\D{D}
        \begin{center}
          \begin{tabular}{|c|c|c|c|c|c|c|c} \hline
                       & \multicolumn{5}{c|}{Number of Ballots} \\
            Ranking     & 4 & 10 & 2 & 3 & 5     \\ \hline\hline  
            1st choice  & \C & \D & \A & \C & \A \\ 
            2nd choice  & \D & \C & \C & \B & \B \\ 
            3rd choice  & \B & \A & \B & \A & \D \\ 
            4th choice  & \A & \B & \D & \D & \C \\ 
            \hline
          \end{tabular}
        \end{center}
        \begin{enumerate}
          \item Does the Borda Count violate the Majority Criterion for this particular preference matrix?
                \solution*{
                  The Borda count tally is
                  \begin{tabular}[c]{|ll|}\hline
                    A & 58 \\
                    B & 46 \\
                    C & 69 \\
                    D & 67 \\ \hline
                  \end{tabular}, \\
                  while a plurality vote yields
                  \begin{tabular}[c]{|ll|}\hline
                    A & 7 \\
                    B & 0 \\
                    C & 7 \\
                    D & 10 \\ \hline
                    total & 24 \\ \hline
                  \end{tabular}
                  \par
                  Thus \fbox{no,} the majority criterion is not violated since no candidate had a majority of the 24 votes.
                  The majority criterion would only be violated if 
                  \emph{both} (a) some candidate had a majority, and (b) the Borda count elected someone different.
                } \vfill          \fillwithlines{\stretch{1}}
          \item Does the Borda Count violate the Head-to-Head Criterion for this particular preference matrix?
                \solution*{
                  \fbox{Yes.} Candidate D would beat each of A, B, and C in head-to-head contests,
                  but the Borda count elects someone else.
                }\vfill           \fillwithlines{\stretch{1}}
        \end{enumerate}

\hwnewpage
  \item Consider the following preference matrix:
        \def\A{A} \def\B{B} \def\C{C} \def\D{D}
        \begin{center}
          \begin{tabular}{|c|c|c|c|c|c|c|c} \hline
                       & \multicolumn{6}{c|}{Number of Ballots} \\
            Ranking     & 11 & 2 & 5 & 1 & 8 & 4     \\ \hline\hline  
            1st choice  & \C & \D & \C & \D & \A & \B \\ 
            2nd choice  & \A & \A & \A & \A & \D & \D \\ 
            3rd choice  & \D & \C & \B & \B & \C & \C \\ 
            4th choice  & \B & \B & \D & \C & \B & \A \\ 
            \hline
          \end{tabular}
        \end{center}
        \begin{enumerate}
          \item Does the Borda Count violate the Majority Criterion for this particular preference matrix?
                \ifsolns
                  \par\soln The Borda count tally is
                  \begin{tabular}[c]{|ll|}\hline
                    A & 93 \\
                    B & 49 \\
                    C & 93 \\
                    D & 75 \\ \hline
                  \end{tabular}
                  \\ while a plurality vote yields
                  \begin{tabular}[c]{|ll|}\hline
                    A & 8 \\
                    B & 4 \\
                    C & 16 \\
                    D & 3 \\ \hline
                    total & 31 \\ \hline
                  \end{tabular}
                  \par
                  Thus \fbox{yes,} the majority criterion is violated, since candidate C has a majority of the 31 votes,
                  but the Borda Count does not elect him; instead, the Borda count has C tying with A.
                \else \vfill          \fillwithlines{\stretch{1}} \fi
          \item Does the Borda Count violate the Head-to-Head Criterion for this particular preference matrix?
                \ifsolns
                  \par\soln
                  \fbox{Yes,} since C beats A, B, and D in head-to-head contests but the Borda Count does not elect him.
                \else \vfill           \fillwithlines{\stretch{1}}\fi
        \end{enumerate}

\hwnewpage
  \item Consider the following preference matrix:
        \def\A{A} \def\B{B} \def\C{C} \def\D{D}
        \begin{center}
          \begin{tabular}{|c|c|c|c|c|c|c|c} \hline
                       & \multicolumn{4}{c|}{\# of Ballots} \\
            Ranking     & 9 & 6 & 8 & 5      \\ \hline\hline  
            1st choice  & \B & \B & \D & \D \\ 
            2nd choice  & \D & \D & \B & \B \\ 
            3rd choice  & \C & \A & \A & \C \\ 
            4th choice  & \A & \C & \C & \A \\ 
            \hline
          \end{tabular}
        \end{center}
        \begin{enumerate}
          \item How many points will each candidate receive in a Borda count?  Who will win?
                \ifsolns
                  \par\soln
                  The point tallies are
                  \begin{tabular}[c]{|ll|}\hline
                    A & 42 \\
                    B & 99 \\
                    C & 42 \\
                    D & 97 \\ \hline
                  \end{tabular}
                  so \fbox{B wins.}
                \else \vfill          \fillwithlines{\stretch{1}}\fi
          \item The five voters in the last column really do think \D\ is the best candidate and \B\ is the second-best.
                However, they decide to be sneaky and lie on their Borda count ballots,
                claiming they think \B\ is the worst candidate;
                in other words, they say they prefer \D, \C, \A, and \B\ in that order.
                Now how many points will each candidate receive in a Borda count?  Who will win?
                \ifsolns
                  \par\soln
                  Now the point tallies become
                  \begin{tabular}[c]{|ll|}\hline
                    A & 47 \\
                    B & 89 \\
                    C & 47 \\
                    D & 97 \\ \hline
                  \end{tabular}
                  so \fbox{D wins.}
                \else \vfill           \fillwithlines{\stretch{1}}\fi
          \item Explain in a few complete sentences how these voters manipulated the Borda count and why it is unfair.
                \ifsolns
                  \par\soln These voters filled out their ballots deceptively, downgrading the candidate they opposed 
                  in order to make their own candidate do well by comparison.
                  This is unfair because it gives dishonest voters more power than honest ones.
                  (Graders, be generous with students' explanations!)
                \else \vfill          \fillwithlines{\stretch{1}}\fi
        \end{enumerate}
				
\hwnewpage
  %COST-CONSCIOUS VOTERS, courtesy of Lisa Moats's REU work
  \item Voters go to the polls to vote on three propositions simultaneously:
        \begin{description}
          \item[Proposition~1.] Spend \$200,000 to plant new flowers and trees at the zoo.
          \item[Proposition~2.] Spend \$400,000 to build a children's playground at the zoo.
          \item[Proposition~3.] Spend \$500,000 to add a panda exhibit to the zoo.
        \end{description}
        All the voters love the zoo, plants, children, and pandas.
        They're all fiscally responsible, and only differ in how much money they want to spend.
        \begin{enumerate}
          \item Alice is one of a thousand voters who want to improve the zoo,
                but don't want to spend more than \$1,000,000 total.
                She wants to spend as close that limit as possible, without going over.
                Which propositions will Alice and her bloc vote for?
                \solution*{Alice et al.~will vote for \fbox{Propositions 2 and 3.}}
          \item Zeke is one of a thousand voters who want to improve the zoo,
                but don't want to spend more than \$800,000 total.
                He wants to spend as close that limit as possible, without going over.
                Which propositions will Zeke and his bloc vote for?
                \solution*{Zeke et al.~will vote for \fbox{Propositions 1 and 3.}}
          \item Shadrach is one of a thousand voters who want to improve the zoo,
                but don't want to spend more than \$600,000 total.
                He wants to spend as close that limit as possible, without going over.
                Which propositions will Shadrach and his bloc vote for?
                \solution*{Shadrach et al.~will vote for \fbox{Propositions 1 and 2.}}
          \item Fill out the following table with the votes---either ``Yes'' or ``No.''\par
                \def\myblank{\raisebox{0pt}[4ex][2ex]{\rule{1in}{0pt}}}
                \ifsolns
                  \def\myblank#1{{\textcolor{red}#1}}
                \else
                  \def\myblank#1{\phantom{X}}
                \fi
                \begin{tabular}{l||c|c|c|}
                  Voters          & Alice et al. & Zeke et al. & Shadrach et al. \\
                  Number of votes & 1000         & 1000        & 1000 \\ \hline\hline
                  Proposition 1   & \myblank N   & \myblank Y  & \myblank Y \\ \hline
                  Proposition 2   & \myblank Y   & \myblank N  & \myblank Y \\ \hline
                  Proposition 3   & \myblank Y   & \myblank Y  & \myblank N \\ \hline
                \end{tabular}
          \item Which of the three propositions will pass?  (Only a majority vote is needed for each proposition.)
                \solution*{\fbox{All three} propositions will pass with 2000 of the 3000 votes.}          \fillwithlines{\stretch{1}}
          \item How many of the 3,000 voters are happy with the result?
                \solution*{Since this will cost taxpayers \$1,100,000, \fbox{none of the voters} will be happy.}          \fillwithlines{\stretch{1}}
          \item Explain in a few complete sentences what is odd or paradoxical about this situation.
                \solution*{Each of the voters deliberately voted to limit spending, 
                  yet they collectively ended up spending more than any of them wanted.
                  \ifgradersolns
                    (Graders, please be generous with their explanations.)
                  \fi
                }          \fillwithlines{\stretch{2}}
        \end{enumerate}
\end{Venumerate}


\ENDHOMEWORK %</HOMEWORK>

%<*WORKSHEETS>



\cleartooddpage

