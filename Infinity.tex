\declareproblemlettering{I}
\pagestyle{fancy}
\cleartooddpage

\chapter{Beyond Numbers}\label{ch:infinity}

%\input{Stories}
\section{Mathematical Stories}\footnote{All stories are quoted from \underline{The Heart of Mathematics} by Edward Burger and Michael Starbird.} \label{sec:Stories}

\newif\ifnudges
%\nudgestrue

\subsection{Damsel in Distress}

Long ago, knights in shining armor battled dragons and rescued damsels in distress on a daily basis. Although it is not often stress the many of the surviving stories of chivalry, frequently the rescue involved logical thinking and creative problem-solving, and often the damsel provided the solution. Here then is a typical knightly encounter.

Once upon a time, a damsel was captured by a notorious knight and imprisoned in a castle surrounded by square moat. The moat was infested with extraordinarily hungry alligators for whom the prospect of a luncheon damsel brought enormous smiles to their green faces. The moat was 20 feet across, and no drawbridge existed because the evil knight took it with him (giving his horse a major hernia).

After a time, a good knight and his squire rode up and said, ``Hail sweet damsel, for I am here; and thou art there; now what are we going to do?''

The knight, though good, was not too bright and the consequently paced back and forth along the moat looking anxiously at the alligators and trying feebly to think of a plan. Then, on the shore the knight found two sturdy beams of wood suitable for walking across but lacking sufficient length. Alas, the moat was 20 feet across, and the beams were each only 19 feet and 8 inches wide. He tried to stretch them and tried to think. Neither effort proved successful. He had no nails, screws, saws, superglue, or any other method of joining the two beams to extend their length.

What to do? What to do? Fortunately, the damsel, after a suitable time to allow the good knight to attempt to solve the puzzle herself, was able to give the knight a few hints that enabled him to rescue her and carried her home to her own castle. How did the maiden advise the knight to accomplish the rescue?

\ifnudges
\hrulefill

Thinking about variations on a situation helps us understand which features are essential and which are unnecessary.  In this case we might consider a variation in which the damsel in distress is on the other side of a 20-foot river rather than surrounded by a square moat.  Unfortunately for the maiden, if she were separated from freedom by a river, she would be stuck because the two 19 foot beams, in the absence of tools, would not enable the knight to rescue her.  Somehow the square shape of the moat must come into play in the solution.

Looking at extremes is a potent technique of analysis in many situations and may be helpful here.  The extremes, ether geometrical ones as in this situation or conceptual extremes in other situations, frequently reveal features that are otherwise hidden.
\fi

\clearpage
\subsection{That's a Meanie Genie}

On an archaeological dig in the Highlands of Tibet, Alley discovered an ancient oil lamp. Just for laughs she rubbed the lamp. She quickly stopped laughing when the huge puff of a magenta smoke sprouted from the lamp, and an ornery genie in Murray appeared. Murray, looking at the stunned Alley, exclaimed, ``Well, what are you staring at? Okay, okay, you've found me; you get your three wishes. So what will they be?'' Alley, although in shock, realized what an incredible opportunity she had. Thinking quickly, she said, ``I'd like to find the Rama Nujan, the jewel that was first discovered by Hardy, the High Lama.'' ``You got it,'' replied Murray, and instantly nine identical looking stones appeared. Alley looked at the stones and was unable to differentiate anyone from the others. 

Finally, she said to Murray, ``So where is the Rama Nujan?'' Murray explained, ``It is embedded in one of the stones. You said you wished to find it. So now you have to find it. Oh, by the way, you may take only one of the stones with you, so you had best be careful how you choose!'' ``But they look identical to me. How will I know which one has the Rama Nujan in it?'' Alley questioned. ``Well, eight of the stones weigh the same, but the stone containing the jewel weighs slightly more than the others,'' Murray responded with a devilish grin.

Alley, now getting annoyed, whispered under her breath, ``Gee, I wish I had a balance scale.'' Suddenly a balance scale appeared. ``That was wish two!'' declared Murray. ``Hey, that's not fair!'' Alley cried. ``You want to talk fair? You think it's fair to be locked in a lamp for 1729 years? You know you can't get internet in there, and there's no room for a satellite dish! So don't talk to me about fair,'' Murray explained. Realizing he had gone a bit overboard, Murray proclaimed, ``Hey, I want to help you out, so let me give you a tip: that balance scale may only be used once.'' ``What? Only once?'' she said, thinking out loud. ``I wish I had another balance scale.'' ZAP! Another scale appeared. ``Okay, kiddo, that was wish three.'' Murray snickered. ``Hey, just one minute,'' Alley said now regretting not having asked for \$1 million or something more standard. ``Well at least this new scale works correctly, right?'' ``Sure, just like the other one. You may use it only once.'' ``Why?'' Alley inquired. ``Because it is a 'wished' balance scale. That means that you can only use it once since it was only one wish. It's just like you cannot wish for 100 more wishes.'' ``You are a very obnoxious genie.'' ``Hey, I don't make up the rules, lady, I just follow them.''

So, Alley may use each of the two balance scales exactly once. Is it possible for Alley to select the slightly heavier stone containing the Rama Nujan stone from among the nine identical looking stones? Please explain why or why not.

\ifnudges
\hrulefill

Initially, we might think that it is impossible to find the jewel since Alley is allowed to make only two comparisons.  Instead of comparing stones individually, perhaps she should compare one collection of stones with another collection of stones.  Now, suppose Alley compares one group with another using the first scale.  What can she conclude?  What should she do next?
\fi
\clearpage
\subsection{The Fountain of Knowledge}

During an incredibly elaborate hazing stunt during pledge week, Trey Sheik suddenly found himself alone in the Sahara desert. His desire to become a fraternity brother was now overshadowed by his desire to find something to drink (these desires, of course, are not unrelated). As he wandered aimlessly through the desert sands, he began to regret his involvement in the whole frat scene. Both hours and miles had passed and Trey was near dehydration. Only now did Trey appreciate the advantages of sobriety. Suddenly, as though it were a mirage, Trey came upon an oasis.

There, sitting in a shaded kiosk beside a small pool of mango nectar, was an old man named Al Donte. Big Al, not only ran the mango bar but was also a travel agent and could book Trey on a two-humped camel back to Michigan. At the moment, however, Trey desired nothing but a large drink of that beautifully translucent and refreshing mangoade. Al informed Trey that the juice was sold only in 8 ounce servings and that the cost for one serving was \$3.50. Trey frantically searched on his pockets and found some change and much sand. Trey counted and discovered that he had exactly \$3.50.

Trey's jubilation at the thought of liquid coating his dried and chapped throat was quickly shattered when Al casually announced that there were no 8- ounce glasses available. Al had only a 6- ounce glass and a 10- ounce glass -- neither of which would have any markings on them. Al, being a man of his word, would not hear of selling any more or any less than an 8- ounce serving of his libation. Trey, in desperation, wondered whether it was possible to use two glasses to produce exactly 8- ounces of mango juice in the 10- ounce glass. Trey thought and thought. Do you think it is possible to use only the unmarked 6- and 10- ounce glasses to produce exactly 8 ounces in the 10 ounce glass? If so, explain how, if not, explain why not.

\ifnudges
\hrulefill

Attempt this puzzle by trial and error together with careful observation.  As we observe the outcomes of various attempts we will teach ourselves what is possible.  Try filling up the 10-ounce glass, and then use it to fill the 6-ounce glass.  What do you now have -- anything new?
\fi

\clearpage
\subsection{Dodge Ball}

%Dodge Ball is a game for two players -- Player One and Player Two (although any two people can play it, even if they are not named ``Player One'' and ``Player Two'').  Each player has his or her own special board and given six turns.
%
%Player One begins by filling in the first horizontal row of their table with a run of X's and O's.  That is, on the first line of his board, he will write six letters -- one in each box -- each letter being either an X or an O.  Then Player Two places either one X or one O in the first box of their board.  So at this point, Player One has filled in the first row of their board with six letters, and Player Two has filled in the first box of their board with one letter.
%
%The game continues with Player One writing down a run of six letters (X's and O's), one in each box of the second horizontal row of their board, followed by Player Two writing one letter (an X or an O) in the second box of their board.  This game proceeds in this fashion until all Player One's boxes are filled with X's and O's; thus, Player One has produced six rows of six marks each, and Player Two has produced one row of six marks.  All marks are visible to both players at all times.  Player One wins if any horizontal row he wrote down is identical to the row that Player Two created (Player One matches Player Two).  Player Two wins if Player Two's string is not one of the six strings made by Player One (Player Two dodges Player One).
%
%Would you rather be Player One or Player Two?  Who has the advantage? Can you devise a strategy for either side that will always result in victory?

Directions for the game:  

Player One begins by filling in the first horizontal row of her table with X's and O's.  Then Player Two places either one X or one O in the first box of his board.  The game continues with Player One writing down a run of six X's and/or O's in the second horizontal row of her board, followed by Player Two writing an X or O in the second box of his board.  The game continues until all boxes on both boards are filled.

	Player One wins if any horizontal row she wrote down is identical to the row that Player Two created (Player One matches Player Two).  Player Two wins if his row is not one of the six rows made by Player One (Player Two dodges Player One).

\noindent Player One's board:
\begin{center}
\renewcommand{\arraystretch}{2}
	\begin{tabular}{|c|*{6}{p{.12\textwidth}|}}
	\hline
1&&&&&&\\	
	\hline
2&&&&&&\\	
	\hline
3&&&&&&\\	
	\hline
4&&&&&&\\	
	\hline
5&&&&&&\\	
	\hline
6&&&&&&\\	 \hline
	\end{tabular}
\end{center}

\noindent Player Two's board:
\begin{center}
\newcolumntype{C}[1]{>{\centering\arraybackslash}p{#1}}
\renewcommand{\arraystretch}{2}
	\begin{tabular}{|*{6}{C{.12\textwidth}|}}
	\hline 1&2&3&4&5&6\\
	\hline &&&&&\\	 \hline
	\end{tabular}
\end{center}

Would you rather be Player One or Player Two?  Who has the advantage?  Why?  Can you devise a strategy for either side that will always result in victory?  This little game holds within it the key to understanding the sizes of infinity. 
\clearpage

\noindent Player One's board:
\begin{center}
\renewcommand{\arraystretch}{2}
	\begin{tabular}{|c|*{6}{p{.12\textwidth}|}}
	\hline
1&&&&&&\\	
	\hline
2&&&&&&\\	
	\hline
3&&&&&&\\	
	\hline
4&&&&&&\\	
	\hline
5&&&&&&\\	
	\hline
6&&&&&&\\	 \hline
	\end{tabular}
\end{center}

\noindent Player Two's board:
\begin{center}
\newcolumntype{C}[1]{>{\centering\arraybackslash}p{#1}}
\renewcommand{\arraystretch}{2}
	\begin{tabular}{|*{6}{C{.12\textwidth}|}}
	\hline 1&2&3&4&5&6\\
	\hline &&&&&\\	 \hline
	\end{tabular}
\end{center}

\vfill

\noindent Player One's board:
\begin{center}
\renewcommand{\arraystretch}{2}
	\begin{tabular}{|c|*{6}{p{.12\textwidth}|}}
	\hline
1&&&&&&\\	
	\hline
2&&&&&&\\	
	\hline
3&&&&&&\\	
	\hline
4&&&&&&\\	
	\hline
5&&&&&&\\	
	\hline
6&&&&&&\\	 \hline
	\end{tabular}
\end{center}

\noindent Player Two's board:
\begin{center}
\newcolumntype{C}[1]{>{\centering\arraybackslash}p{#1}}
\renewcommand{\arraystretch}{2}
	\begin{tabular}{|*{6}{C{.12\textwidth}|}}
	\hline 1&2&3&4&5&6\\
	\hline &&&&&\\	 \hline
	\end{tabular}
\end{center}
\clearpage
\subsection{Dot of Fortune}

One day three college students were selected at random from the studio audience to play the ever-popular TV game show, ``Dot of Fortune.'' One of the students already had discovered the power and beauty of mathematical thinking, while the other two were not nearly so fortunate.  The stage had no mirrors, reflecting surfaces, or television monitors.  The three students were led blindfolded to their places around a small round table.  As the rules of the game were explained by Pat, Vanna affixed to each of the three youthful foreheads a conspicuous but small colored dot.

``So, contestants,'' Pat explained, ``at the bell your blindfolds will be removed.  You will see your two companions sitting quietly at the table, each with a dot on his or her forehead.  Each dot is either red or white.  You cannot, of course, see the dot on your own forehead.  After you have observed the dots on your companions' foreheads, you will raise your hand if you see at least one red dot. If you do not see a red dot, you will keep your hands on the table. The object of the game is to deduce the color of your own dot.  As soon as you know the color of your dot, you are to hit the buzzer in front of you.  Do you understand the rules of the game?'' All the students understood the rules, although the math fan understood them better.

``Are you ready?'' asked Vanna after affixing three red dots to the foreheads of the three students.  After the three contestants nodded, Vanna instructed them to simultaneously remove their blindfolds as the studio audience quivered with anticipation.  The three students looked at one another's dots, and all raised their hands.  After some time, the math fan hit her buzzer knowing what color dot she had.  Please explain how she knew this.  Why did the other students not know?

\ifnudges
\hrulefill

Sometimes no action is action enough.  Put yourself in the position of one of the three contestants.  You know that the dot on your forehead is either red or white.  The trick to figuring out this conundrum is to suppose you have a white dot and see what would happen.

Suppose you are sitting at the table looking at two red dots, and you assume that you have a white dot on your forehead.  What would each of the two others at the table see?  What could they deduce?  What would they do?  What did they do? What can you conclude?
\fi

\clearpage

%\input{Correspondence}
\section{What does ``2'' mean?} \label{sec:correspondence}

Our journey to infinity begins with 2. We are all familiar with 2, and we will use that intuitive
intimacy to develop a concept of size that will take us well beyond what we know.

If we have two apples and two hands, we could put one apple in each hand, and no hand would be un-appled, or any apple un-handed. If we have two socks and we put one sock on each hand, then our socks and
our hands correspond exactly. This observation also implies that we can put one apple in each sock and demonstrate that the socks and apples also correspond; however, we do not recommend this last experiment unless the socks are clean.

\boxedblank{\textbf{One-to-one Correspondence:}}

Grappling with infinity requires us to imagine physical scenarios that are not quite possible but can be clearly conceived in the mind. Suppose we have a huge barrel full of Volvos and a barrel full of soccer balls, and we want to know if there are more Volvos than soccer balls, more soccer
balls than Volvos, or the same number of each. How would we decide?

We could take one Volvo from the first barrel and one soccer ball from the second, pair them up and put them aside (probably putting the ball in the car's trunk). Then we could pair up another Volvo from the first barrel with another soccer ball from the second. If we continue pairing in this way and every car has a ball and every ball has a car, we know there are the same number of each.

This simple idea is important. We have just described a method for determining when two collections have the same number of objects in them without actually naming that number. Two collections whose objects can be paired together evenly-one from one collection with one from the other collection-are said to have a \textbf{one-to-one correspondence}.
\clearpage

\begin{enumerate}
\item  \textbf{The same, but unsure how much}. We have used a method of checking whether two sets of objects have the same number of things by pairing up and removing one object from each set until we run out. If we run out of objects from both sets at the same time, then we know that the sets contain the same number of things. Otherwise, we know that one set is larger than the other. Describe several events whereby we are able to compare the size of two collections without computing the individual sizes-for example, people filling all seats in an auditorium.

%\vfill \item  \textbf{Taking stock}. It turns out that there is a one-to-one correspondence between the New York Stock Exchange symbols for companies and the companies themselves (for example, PE is Philadelphia Electric Company). Explain why this correspondence must be one-to-one. What would happen if it were not? Describe potential problems.

\vfill \item  \textbf{Don't count on it}. The following are two collections of the symbols @ and \copyright.
\begin{center} \small
	\begin{tabular}{*{19}{c}}
@  & @  & @  & @  & @  & @  & @  & @  & @  & @  & @  & @  & @  & @  & @  & @  & @  & @  & @ \\
\copyright  & \copyright  & \copyright  & \copyright  & \copyright  & \copyright  & \copyright  & \copyright  & \copyright  & \copyright  & \copyright  & \copyright  & \copyright  & \copyright  & \copyright  & \copyright  & \copyright  & \copyright  & \copyright \\
	\end{tabular}
\end{center}
Are there more @'s than \copyright's? Describe how you can quickly answer the question without counting, and explain the connection with the notion of a one-to-one correspondence.

\vfill \item \textbf{Here's looking @  \textregistered} Below are two collections of the symbols @  and \textregistered.
\begin{center} \small
	\begin{tabular}{*{19}{c}}
@  &  @  &  @  &  @  &  @  &  @  &  @  &  @  &  @  &  @  &  @  &  @  &  @  &  @  &  @  &  @  &  @  &  @  &  @ \\
\textregistered  &  \textregistered  &  \textregistered  &  \textregistered  &  \textregistered  &  \textregistered  &  \textregistered  &  \textregistered  &  \textregistered  &  \textregistered  &  \textregistered  &  \textregistered  &  \textregistered &\textregistered  &  \textregistered  &  \textregistered \\ &  &  & 
	\end{tabular}
\end{center}
Are there more @'s than \textregistered's? Describe how you can quickly answer the question without actually counting, and explain the connection with the notion of a one-to-one correspondence. 
\vfill 

%\clearpage
%\item \textbf{Enough underwear}. When Deb packs for a trip, she doesn't count how many days she is going to be away and then count out that many pairs of underwear. Instead she places underwear into her suitcase, one at a time, and says the name of each day she will be away: ``Monday'' (and places one in), ``Tuesday'' (and places another in), ``Wednesday'', etc. Using this method, does Deb know the number of underwear she placed in her bag? Does she have enough underwear for her trip? Discuss the connection this true story has with the notion of a one-to-one correspondence. 
%
%\vfill \item \textbf{ 791ZWV}. Suppose a stranger tells you that the license plate number on his car is 791ZWV. IF you had a listing of all automobiles in the United States together with their license plate numbers, would you be able to precisely identify the stranger's vehicle? If so, explain why. If not, explain why, and identify what additional information you would require to identify it exactly. Discuss the connection between this situation and the notion of a one-to-one correspondence. 
%
%\vfill \item \textbf{245-2345}. Suppose a stranger tells you that her telephone number is 245-2345. Would you be able to dial her number and be certain that you reach her? If so, explain why. If not, explain why, and identify what additional information you would require to be certain to reach her home. Discuss the connection between this situation and the notion of a one-to-one correspondence.
%
%\vfill \item \textbf{Social Security}. Is there a one-to-one correspondence between U.S. residents and their social security numbers? Explain why or why not.

%\vfill \item \textbf{ Testing one two three}. A professor wishes to distribute one examination to each student in the class. What is the most efficient way for her to determine whether she has more students than exams:  Pass the exams out or count?
%
%\vfill \item \textbf{Laundry day}. Suppose you are given a bag of quarters. The laundry machine requires \$1.75 worth of quarters. One way to count how many washes you can do is to take out one quarter and say 25\textcent, then take out another and say 50\textcent, and so on. In practice, however, you might well use a different method that uses a notion of one-to-one correspondence. Explain such a method.

\vfill 

%\clearpage
\item \textbf{Hair counts}. Do there exist two nonbald people on Earth having the property that there is a one-to-one correspondence between the collection of hairs on one person's body and the collection or hairs on the other person's body? 

\vfill \item \textbf{Social Security number}. A social security number is a nine-digit number. Suppose that all nine-digit numbers are allowable social security numbers. Is there a one-to-one correspondence between allowable social security numbers and U.S. residents? You may assume that the U.S. population is about 250 million. Explain your answer.

%\vfill \item \textbf{Musical chairs}. Musical chairs is a fun game in which a group of people march around a row of chairs while music is played. There is one more person than there are chairs. The moment the music stops, everyone scrambles for a chair. The person left chairless loses and moves off to the sidelines. Then everyone in a chair gets up, one chair is removed, and the music and marching begin again. At what points in this game do we have a one-to-one correspondence between chairs and people and at what points do we not have such a correspondence? Explain the correspondences as the game is played until there is a winner.

%\vfill \item \textbf{Dining hall blues}. One day during dinner in Ralph P. Uke Dining Hall the students in line for food discovered that the dining hall had run out of forks (no clean or dirty ones were available). While there was much jubilation in the line, what can you conclude about the set of all students in the dining hall and the set of all forks? Do we know how many forks there are? How many students? Discuss how the issue of one-to-one correspondence is relevant in addressing these questions.

\vfill \item \textbf{Dorm life}. Every student at a certain college is assigned to one and only one dorm room. Does this imply that there is a one-to- one correspondence between dorm rooms and students? Explain your answer. \vfill
\end{enumerate}

\clearpage
\section{Familiar, but Infinite} \label{Infinite}

What is familiar and concrete to one person may be foreign and abstract to another, but as far as numbers are concerned we probably all agree on which are the most familiar. In 1886 Leopold Kronecker, a number theorist, made a statement about what is basic in the world of mathematics: ``God created the positive integers; all the rest is human creation.''

One, two, three, .. . these are the positive integers. For every positive integer there is a next bigger one. Although we may think of these positive integers successively, we may also think of all of them at once, that is, think about the collection of all positive integers. The collection (also referred to as the set) of positive integers is so basic and natural to our way of thinking that it is called the set of natural numbers.

The set of all natural numbers is our first infinite set, and it has a comfortable feel about it. Among infinite sets, the natural numbers seem the most natural. By examining this and related collections of numbers, we will begin to develop a better and more precise idea of infinity.


\noindent \textbf{Question:} Are there as many natural numbers as there are integers starting with 2, 3, 4, .. . , we are asking neither more nor less than the question: Is there a one-to-one correspondence
between the elements of the set 
\[2, 3, 4, .. .\]
and the set of natural numbers
\[1, 2, 3, .. . ?\]
\clearpage

\begin{enumerate}
	%\item \textbf{Even odds}. Let $E$ stand for the set of all even natural numbers (so $E = (2, 4, 6, 8, ... . )$ ) and $O$ stand for the set of all odd natural numbers (so $O = (1, 3, 5, 7, .... )$). Show that the sets $E$ and $O$ have the same cardinality by desc1ibing an explicit one-to-one correspondence between the two sets.
%\vfill 

\item \textbf{Naturally even}. Let $E$ stand for the set of all even natural numbers (so $E = (2, 4, 6, 8, .... )$). Show that the set $E$ and the set of all natural numbers have the same cardinality by describing an explicit one-to-one correspondence between the two sets.
\vfill 

\item \textbf{5's take over}. Let \textbf{EIF} be the set of all natural numbers ending in 5 (\textbf{EIF} stands for "ends in five"). That is,
\[\mathbf{EIF} = (5, 15, 25, 35, 45, 55, 65, 75, . . . ).\]
Describe a one-to-one correspondence between the set of natural numbers and the set \textbf{EIF}.
\vfill 

\item \textbf{6 times as much}. If we let $\mathbb{N}$ stand for the set of all natural numbers, then we write $6\mathbb{N}$ for the set of natural numbers all multiplied by 6 (so $6\mathbb{N} = (6, 12, 18, 24, . .. )$). Show that the sets $\mathbb{N}$ and $6\mathbb{N}$ have the same cardinality by describing an explicit one-to-one correspondence between the two sets.
\vfill 

\item \textbf{Any times as much}. If we let $\mathbb{N}$ stand for the set of all natural numbers, and $a$ stand for any particular natural number, then we write $a\mathbb{N}$ for the set of natural numbers all multiplied by $a$. Do the sets $\mathbb{N}$ and $a\mathbb{N}$ have the same cardinality? If so, describe an explicit one-to-one correspondence between the two sets.
\vfill 

%\clearpage
%\item  \textbf{Missing 3}. Let \textbf{TIM} be the set of all natural numbers except the number 3 (\textbf{TIM} stands for ``three is missing''), so $TIM = (1, 2, 4, 5, 6, 7, 8, 9, ... )$. Show that the set TIM and the set of all natural numbers have the same cardinality by describing an explicit one-to-one correspondence between the two sets.
%\vfill 
%
%\item \textbf{One weird set}. Let \textbf{OWS} (you figure it out) be the set defined by 
%\[OWS = (1, 3, 5, 7, 8, 10, 11, 12, 13, 14, 15, 16, ... );\]
 %that is, after 8, the set contains all the natural numbers from 10 on. Show that the set \textbf{OWS} and the set of all natural numbers have the same cardinality by describing an explicit one-to-one correspondence between the two sets.
%
%\vfill 

\item \textbf{Reciprocals}. Suppose $R$ is the set defined by 
\[R = 1/1, 1/2, 1/3, 1/4, 1/5, 1/6, ... .\]
Describe the set $R$ in words. Show that it has the same cardinality as the set of natural numbers.

\vfill 

%\item  \textbf{Hotel Cardinality (formerly California)}. It is the stranded traveler's fantasy. The Hotel Cardinality is a full-service luxury hotel with bar and restaurant, having as many rooms as there are natural numbers. The room numbers are 1, 2, 3, 4, 5, .... You can see why stranded travelers love the Hotel Cardinality. There appears to be no need for the sad sign: ``No Vacancy.'' Suppose, however, that every room is completely filled. Now it appears that the night manager must flash the No Vacancy sign. What if a wealthy traveler arrives late in the night looking for a place to stay? Could the night manager figure out a way to give the traveler her own private room (no sharing) without having anyone leave the hotel? The answer is yes. Describe how this accommodation can be made; of course some guests will have to move to other rooms. 
%
 %\vfill 

\item \textbf{Half way}. Suppose you take the line below,

\hrulefill

cut it in half, and then take the left piece and cut it in half, and then take the leftmost piece and cut it in half, and so on, without ever stopping. How many different pieces of the line would you have? Does the set of all pieces have the same cardinality as the set of all natural numbers? Justify your answer. \vfill
\end{enumerate}

\clearpage
\section{Rational Numbers} \label{RationalNumbers}

We might feel more or less stymied in our quest for an infinite set bigger than the set of natural numbers. But let's not forget that lots of numbers are not integers. For example, how about the rational numbers? Recall that the set of rational numbers is the set of all ratios of integers (fractions). Between any two integers there are infinitely many rationals. Within the set of rational numbers, we actually have infinitely many different infinite sets. The set of rational numbers must be huge.

Unfortunately, in our search for different sizes of infinity, the rationals are still not numerous enough. We rely once again on the definition of same cardinality; namely, two sets have the same cardinality if there is a one-to-one correspondence between the elements of one set and the elements of the other. The trick here is to write down the rational numbers in a convenient and systematic way so that we know we have listed them all. The idea is to put all the rational numbers with numerator 1 in one column, all those with numerator 2 in another column, and so on as shown in the following diagram. Notice that all the rationals in the same row have the same denominator. For example, to find 37/112 in the diagram, we just go 37 spaces to the right of 0 and 112 spaces up. So, we can see that all the rational numbers appear somewhere in the pattern. In fact, each rational number appears many times, because the fractions are not all reduced to lowest terms. For example, 1/2 appears and so does 2/4, 3/6, and so on, but that redundancy is okay, because at this point we merely want a systematic method of writing down every rational number without leaving any out. Notice that all the positive rational numbers appear in the upper-right part of the diagram, all the negative rationals appear in the lower left, and 0 is right in the middle. So far, then, we have described a way of writing down all the rational numbers.

\begin{center}
	{ \small
\begin{tabular}{*{13}{c}}
 &  &  &  &  &  &  & $\vdots$ & $\vdots$ & $\vdots$ & $\vdots$ & $\vdots$ & \\
 &  &  &  &  &  &  & 1/5 & 2/5 & 3/5 & 4/5 & 5/5 & $\cdots$\\
 &  &  &  &  &  &  & 1/4 & 2/4 & 3/4 & 4/4 & 5/4 & $\cdots$\\
 &  &  &  &  &  &  & 1/3 & 2/3 & 3/3 & 4/3 & 5/3 & $\cdots$\\
 &  &  &  &  &  &  & 1/2 & 2/2 & 3/2 & 4/2 & 5/2 & $\cdots$\\
 &  &  &  &  &  &  & 1/1 & 2/1 & 3/1 & 4/1 & 5/1 & $\cdots$\\
 &  &  &  &  &  & 0 &  &  &  &  &  & \\
$\cdots$ & -5/1 & -4/1 & -3/1 & -2/1 & -1/1 &  &  &  &  &  &  & \\
$\cdots$ & -5/2 & -4/2 & -3/2 & -2/2 & -1/2 &  &  &  &  &  &  & \\
$\cdots$ & -5/3 & -4/3 & -3/3 & -2/3 & -1/3 &  &  &  &  &  &  & \\
$\cdots$ & -5/4 & -4/4 & -3/4 & -2/4 & -1/4 &  &  &  &  &  &  & \\
$\cdots$ & -5/5 & -4/5 & -3/5 & -2/5 & -1/5 &  &  &  &  &  &  & \\
 & $\vdots$ & $\vdots$ & $\vdots$ & $\vdots$ & $\vdots$ &  &  &  &  &  &  & 
\end{tabular}}
\end{center}
To show the one-to-one correspondence between the rational numbers and the natural numbers, we will thread a single rectangular spiral through all the rationals, starting in the middle at 0 and moving counterclockwise outward. To see the one-to-one correspondence with the natural numbers, we will just count the rational numbers as we encounter them along the spiral and make them bold-face to remind us that we have paired that rational with some natural number. We start with the rational O corresponding to the natural number l ; then, moving to the nght and up, the rational $1/1 = 1$ corresponds to the natural number 2, the rational $-1/1 = - 1$ corresponds to 3, the rational $2/1 = 2$ corresponds to 4. We next come to 2/2, which has already been counted, so we skip it and move to 1/2, which corresponds to 5, then -2/1 = -2 corresponds to 6. We skip -2/2 since that equals - 1, which already corresponds to 3, and move to -1/2, which corresponds to 7, and so on. Notice that eve1y rational number will eventually be reached and put in correspondence with some natural number. This one-to-one correspondence shows that the set of all rational numbers has the same cardinality as the set of the natural numbers.  

Threading the spiral and counting along it provides an important insight into sets with the same cardinality as the set of natural numbers. If we can write a set out as an infinite list, we can make a one-to-one correspondence with the natural numbers.

We now see that the rational numbers did not provide us with an infinity larger than that of the natural numbers. Our quest for an even grander infinity has thus far failed. But perhaps when we are chasing sets larger than an infinite set, we should expect to have to go a long, long way.

%\clearpage
\begin{enumerate}
	\item \textbf{A grand union}. Suppose you have two sets, each having the same
cardinality as the set of natural numbers. Take the elements of both
sets and put them together to make one huge set. Prove that this
new huge set has the same cardinality as the set of natural numbers. \vfill
\item \textbf{ Unnoticeable pruning}. Suppose you have any infinite set. Is it
always possible to remove some things from that set such that the
collection of remaining things has the same cardinality as the original
set? Explain why or why not, and illustrate your answer with an
example.\vfill
\end{enumerate}

\clearpage

%\section{The Missing Member}  \label{sec:MissingMember}\solnsfalse

Around 1872 the German mathematician Georg Cantor shook the foundations of infinity when he showed that the set of real numbers has more elements than the set of natural numbers. In other words, he proved that infinity is not one size but that some infinities are more infinite than others. At first such a notion seems almost nonsensical. Once we have reached infinity, surely we cannot climb farther. But Cantor showed that there were yet higher mountains to scale.

To show that the real numbers are more numerous than the natural numbers, Cantor focused intently on what it would mean for the real numbers and the natural numbers to have the same cardinality. It would mean that the real and natural numbers could be put in one-to-one correspondence. Writing down what such a correspondence might look like gives us a visual clue how to demonstrate conclusively that any attempted correspondence between the natural numbers and the real numbers could not include every real; some real is missing-\emph{the missing member}.

To figure out Cantor's argument, we need to recall that each real number can be expressed as an infinitely long decimal expansion. For example,
\[ 243.4 76666875446800887672875849345788445321 \dots\]
is a real number. Before moving forward, we must first make an easy observation about real numbers. Suppose we examine two decimal numbers, but we cover up all the digits in the numbers with ?s except for the digit that is in, say, the fifth place after the decimal point. So, we have a piece of paper with two funny looking numbers on 1't : $??.????2????\dots$ and $??.????4????\dots$. We do not know what these numbers are because we can read only the fifth digit after the decimal point. But one thing we do know is that these two numbers are different. If they were the same, we could not have a 2 in the fifth place after the decimal point of one number and a 4 in the fifth place in the other. Likewise, if we have two numbers and one has a 2 and the other has a 4 in the 87th place after the decimal, then the two numbers must be different. This observation is not hard to understand, but it is a key to Cantor's reasoning.  

Cantor proved that there are more real numbers than natural numbers through a clever, yet simple idea. If the set of real numbers and the set of natural numbers had the same cardinality, then there would be a one-to- one correspondence between the set of natural numbers and the set of real numbers. So his idea was to list the natural numbers down the left-hand side of a page, list reals in the right-hand column, and then show how to construct a real number that could not appear on the list. He showed that, once we commit ourselves to a list of reals in the right-hand column, one real number corresponding to each natural number, then we can describe a real decimal number that does not appear anywhere on that infinite list. So, we \emph{could not} have listed all the real numbers in the right-hand column. Thus, the natural numbers and the real numbers could not be put in one-to-one correspondence, and so there are more real numbers than natural numbers. Cantor's basic strategy was to attempt an impossible task in order to understand why it couldn't be done. 

We are going to write down a particular real number that we will call $M$, for "missing." We will w1ite it in its decimal expansion. Our number $M$ will be between 0 and 1, so its decimal expansion begins with 0.???\dots. Now we must decide what the digits ''???\dots'' are. Each digit will be one of two possibilities: a 2 or a 4. We will decide on the digits of our number $M$ one at a time, successively, so we must be patient. We now describe the criterion by which we choose each digit of our number $M$.

\ifsolns
We start with the first digit after the decimal point. Remember that we have a table that pairs one real number with each natural number. So some real number is paired with the natural number 1. We take a look at that real number and look at its first digit after the decimal point. Although this insight will not shake the ve1y foundations of your universe, we boldly state that there are only two possibilities for that first digit: It is either 2 or it is not 2. We will use the first digit of that first real number to decide on the first digit of our number $M$- the real number we are building. If the first digit of that first real number is 2, then we will set the first digit after the decimal point of our number $M$ to be 4. If, however, the first digit of that first real number is not 2, then we will set the first digit after the decimal point of our number $M$ to be 2. Observe that, no matter what digits come next in $M$, we know for sure that the number $M$ will not equal the real number paired with 1. Why? Because $M$ and the real number paired with 1 have different first digits after the decimal point!

How will we define the digit that is in the second place after the decimal point of our number $M$? We take a look at the real number paired with the natural number 2, see what its second place digit after the decimal point is, and ask if it equals 2. If that digit is 2, then we will set the second digit of our number $M$ to be 4. If, however, that digit is not 2, we will set the second digit of our number $M$ to be 2. Notice that we have defined $M$ such that $M$'s second digit after the decimal point is not the same as the second digit after the decimal point of the real number corresponding to 2. In particular, $M$ cannot equal the second real number in the list- the real number corresponding to 2.

We continue to define the digits of $M$ in this fashion. So, for example, to determine the 11th digit of $M$, we look at the 11th digit in the real number that is paired with the natural number 11. If that digit is 2, then we define the 11th digit of our number $M$ to be 4; if that digit is not 2, then we define the 11th digit of our number $M$ to be 2.
\else
\begin{center}
	\begin{tabular}{cc}
	1 & 0.76206658\\ \hline
2 & 0.910928721\\ \hline
3 & 0.055783247\\ \hline
4 & 0.615411668\\ \hline
5 & 0.520502311\\ \hline
6 & 0.782172298\\ \hline
7 & 0.874793025\\ \hline
8 & 0.334273456\\ \hline
	\end{tabular}
\end{center}
\vfill
\begin{center}
\newcolumntype{C}[1]{>{\centering\arraybackslash}p{#1}}
\renewcommand{\arraystretch}{2}
	\begin{tabular}{*{9}{C{.05\textwidth}}}
	\hline &1&2&3&4&5&6&7&8\\
	\hline 0.&&&&&&&\\	 \hline
	\end{tabular}
\end{center}
\vfill
\fi

Is $M$ on our list of real numbers?
\clearpage

\begin{enumerate}
	\item \textbf{Don't dodge the connection}. Explain the connection between the Dodge Ball game and Cantor's proof that the cardinality of the reals is greater than the cardinality of the natural numbers. \vfill

\item \textbf{Cantor with 3's and 7's}. Rework Cantor's proof from the beginning but this time, if the digit under consideration is 3, then make the corresponding digit of $M$ a 7, and if the digit is not 3, make the associated digit of $M$ a 3. \vfill

\item \textbf{Think positive}. Prove that the cardinality of the positive real numbers is the same as the cardinality of the negative real numbers. (Caution: You need to describe a one-to-one correspondence; however, remember that you cannot list the elements in a table .) \vfill

\item \textbf{Diagonalization}. Cantor's proof is often referred to as ``Cantor's diagonalization argument.'' Explain why this is a reasonable name. \vfill

\item  \textbf{No Vacancy}. Recall the Hotel Cardinality, described in the previous section. Create a collection of people so that it would be impossible for the (new) night manager to give each of them a room. Thus, for a really big group of people, a No Vacancy sign (or actually a Not Enough Room sign) might actually be necessary. Explain why it is not possible to give each person from your group a room. \vfill

\item \textbf{Just guess}. This is just a ``guessing question.'' Do you think there are sets whose cardinality is actually larger than that of the set of real numbers? Or do you think the infinity of reals is the largest infinity? Just make a guess and informally explain it. \vfill 

\end{enumerate}
\section{The Missing Member}  \label{sec:MissingMember}\solnsfalse

Around 1872 the German mathematician Georg Cantor shook the foundations of infinity when he showed that the set of real numbers has more elements than the set of natural numbers. In other words, he proved that infinity is not one size but that some infinities are more infinite than others. At first such a notion seems almost nonsensical. Once we have reached infinity, surely we cannot climb farther. But Cantor showed that there were yet higher mountains to scale.

To show that the real numbers are more numerous than the natural numbers, Cantor focused intently on what it would mean for the real numbers and the natural numbers to have the same cardinality. It would mean that the real and natural numbers could be put in one-to-one correspondence. Writing down what such a correspondence might look like gives us a visual clue how to demonstrate conclusively that any attempted correspondence between the natural numbers and the real numbers could not include every real; some real is missing-\emph{the missing member}.

To figure out Cantor's argument, we need to recall that each real number can be expressed as an infinitely long decimal expansion. For example,
\[ 243.4 76666875446800887672875849345788445321 \dots\]
is a real number. Before moving forward, we must first make an easy observation about real numbers. Suppose we examine two decimal numbers, but we cover up all the digits in the numbers with ?s except for the digit that is in, say, the fifth place after the decimal point. So, we have a piece of paper with two funny looking numbers on 1't : $??.????2????\dots$ and $??.????4????\dots$. We do not know what these numbers are because we can read only the fifth digit after the decimal point. But one thing we do know is that these two numbers are different. If they were the same, we could not have a 2 in the fifth place after the decimal point of one number and a 4 in the fifth place in the other. Likewise, if we have two numbers and one has a 2 and the other has a 4 in the 87th place after the decimal, then the two numbers must be different. This observation is not hard to understand, but it is a key to Cantor's reasoning.  

Cantor proved that there are more real numbers than natural numbers through a clever, yet simple idea. If the set of real numbers and the set of natural numbers had the same cardinality, then there would be a one-to- one correspondence between the set of natural numbers and the set of real numbers. So his idea was to list the natural numbers down the left-hand side of a page, list reals in the right-hand column, and then show how to construct a real number that could not appear on the list. He showed that, once we commit ourselves to a list of reals in the right-hand column, one real number corresponding to each natural number, then we can describe a real decimal number that does not appear anywhere on that infinite list. So, we \emph{could not} have listed all the real numbers in the right-hand column. Thus, the natural numbers and the real numbers could not be put in one-to-one correspondence, and so there are more real numbers than natural numbers. Cantor's basic strategy was to attempt an impossible task in order to understand why it couldn't be done. 

We are going to write down a particular real number that we will call $M$, for "missing." We will w1ite it in its decimal expansion. Our number $M$ will be between 0 and 1, so its decimal expansion begins with 0.???\dots. Now we must decide what the digits ''???\dots'' are. Each digit will be one of two possibilities: a 2 or a 4. We will decide on the digits of our number $M$ one at a time, successively, so we must be patient. We now describe the criterion by which we choose each digit of our number $M$.

\ifsolns
We start with the first digit after the decimal point. Remember that we have a table that pairs one real number with each natural number. So some real number is paired with the natural number 1. We take a look at that real number and look at its first digit after the decimal point. Although this insight will not shake the ve1y foundations of your universe, we boldly state that there are only two possibilities for that first digit: It is either 2 or it is not 2. We will use the first digit of that first real number to decide on the first digit of our number $M$- the real number we are building. If the first digit of that first real number is 2, then we will set the first digit after the decimal point of our number $M$ to be 4. If, however, the first digit of that first real number is not 2, then we will set the first digit after the decimal point of our number $M$ to be 2. Observe that, no matter what digits come next in $M$, we know for sure that the number $M$ will not equal the real number paired with 1. Why? Because $M$ and the real number paired with 1 have different first digits after the decimal point!

How will we define the digit that is in the second place after the decimal point of our number $M$? We take a look at the real number paired with the natural number 2, see what its second place digit after the decimal point is, and ask if it equals 2. If that digit is 2, then we will set the second digit of our number $M$ to be 4. If, however, that digit is not 2, we will set the second digit of our number $M$ to be 2. Notice that we have defined $M$ such that $M$'s second digit after the decimal point is not the same as the second digit after the decimal point of the real number corresponding to 2. In particular, $M$ cannot equal the second real number in the list- the real number corresponding to 2.

We continue to define the digits of $M$ in this fashion. So, for example, to determine the 11th digit of $M$, we look at the 11th digit in the real number that is paired with the natural number 11. If that digit is 2, then we define the 11th digit of our number $M$ to be 4; if that digit is not 2, then we define the 11th digit of our number $M$ to be 2.
\else
\begin{center}
	\begin{tabular}{cc}
	1 & 0.76206658\\ \hline
2 & 0.910928721\\ \hline
3 & 0.055783247\\ \hline
4 & 0.615411668\\ \hline
5 & 0.520502311\\ \hline
6 & 0.782172298\\ \hline
7 & 0.874793025\\ \hline
8 & 0.334273456\\ \hline
	\end{tabular}
\end{center}
\vfill
\begin{center}
\newcolumntype{C}[1]{>{\centering\arraybackslash}p{#1}}
\renewcommand{\arraystretch}{2}
	\begin{tabular}{*{9}{C{.05\textwidth}}}
	\hline &1&2&3&4&5&6&7&8\\
	\hline 0.&&&&&&&\\	 \hline
	\end{tabular}
\end{center}
\vfill
\fi

Is $M$ on our list of real numbers?
\clearpage

\begin{enumerate}
	\item \textbf{Don't dodge the connection}. Explain the connection between the Dodge Ball game and Cantor's proof that the cardinality of the reals is greater than the cardinality of the natural numbers. \vfill

\item \textbf{Cantor with 3's and 7's}. Rework Cantor's proof from the beginning but this time, if the digit under consideration is 3, then make the corresponding digit of $M$ a 7, and if the digit is not 3, make the associated digit of $M$ a 3. \vfill

\item \textbf{Think positive}. Prove that the cardinality of the positive real numbers is the same as the cardinality of the negative real numbers. (Caution: You need to describe a one-to-one correspondence; however, remember that you cannot list the elements in a table .) \vfill

\item \textbf{Diagonalization}. Cantor's proof is often referred to as ``Cantor's diagonalization argument.'' Explain why this is a reasonable name. \vfill

\item  \textbf{No Vacancy}. Recall the Hotel Cardinality, described in the previous section. Create a collection of people so that it would be impossible for the (new) night manager to give each of them a room. Thus, for a really big group of people, a No Vacancy sign (or actually a Not Enough Room sign) might actually be necessary. Explain why it is not possible to give each person from your group a room. \vfill

\item \textbf{Just guess}. This is just a ``guessing question.'' Do you think there are sets whose cardinality is actually larger than that of the set of real numbers? Or do you think the infinity of reals is the largest infinity? Just make a guess and informally explain it. \vfill 

\end{enumerate}