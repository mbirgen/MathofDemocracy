\section{Savings} \index{annuity} \index{future value} \label{sec:Savings}



  %%%%%\item Deriving a very useful formula:
	%%%%%\ifsolns
		%%%%%The basic idea for a future value annuity is that every month we receive compound interest on our new
%%%%%payment along with all of our previous payments. Therefore at each time when we are going to
%%%%%compound the interest, each payment has been compounded an additional time.
%%%%%
%%%%%Example: Payment  is paid four times a year and the interest is compounded quarterly for an annual
%%%%%interest rate of  for one year:
%%%%%\begin{itemize}
	%%%%%\item  Start: 0
%%%%%\item After quarter one: $R$
%%%%%\item After quarter two the original $R$ is compounded once now and we pay an additional $R$
%%%%%\[R (1 +\frac r4)
 %%%%%+ R\]
%%%%%\item After quarter three the original $R$ has been compounded twice, the second $R$ has been
%%%%%compounded once, and we pay an additional $R$
%%%%%\[
%%%%%R(1+\frac r4)^2	
%%%%%+ R (1 +\frac r4)
%%%%%+ R\]
%%%%%\item After quarter four (one year) the original $R$ has been compounded three times, the second $R$ has
%%%%%been compounded twice, the third $R$ has been compounded once, and we pay an additional $R$
%%%%%\[ R(1+\frac r4)^3+
%%%%%R(1+\frac r4)^2
%%%%%+ R (1 +\frac r4)
%%%%%+ R\]
%%%%%\item Thus you can see the pattern that the original payment will always be compounded one less that
%%%%%compounding period, which in general will be $nt$ for $n$ compounds per year and $t$ total years
%%%%%
%%%%%\end{itemize}
%%%%%We can now see t
The general formula for Future Value $F$ is
\begin{align}
F=&R(1 +\frac rn)^{nt-1}+R(1 +\frac rn)^{nt-2}+\cdots +R(1 +\frac rn)^1+R\\
	=& R\frac{(1+\frac rn)^{nt}-1}{\frac rn}
	\end{align}
\fi
        \fillwithlines{\stretch{1}}
				\begin{enumerate}
  \item An \defnstyle{annuity} is a series of equal payments made at regular intervals.
        For an \defnstyle{ordinary annuity}, the payments are made at the \emph{end} of the interval. \index{annuity!ordinary}
        If the payments are made at the \emph{beginning} of the interval, it is called an \defnstyle{annuity due}.  We will not be working with this type of annuity in this class.\index{annuity!due}
				
				A good way to think about these is as a savings account.  You are putting some money, $R$, into the account every compounding period in order to save up for something in the future.
        \fillwithlines{\stretch{1}}
\clearpage
  \item \boxedblank[2in]{\textbf{Future Value of an Annuity:} \par \color{magenta} \[F=R\left(\frac{(1+\frac rn)^{nt}-1}{\frac rn}\right)\]
	If we let $i=\frac{r}{n}$ and $m=nt$, we can shorten the formula to:
	\[ F=R\left(\frac{(1+i)^{m}-1}{i}\right)\]}
%  \item Notes on the time value of money:
%        \vfill
  \item \textbf{Example:}
        You graduate, get a job, and want to buy a house in two years.
        How much should you deposit each month into an account bearing 7\% interest, compounded monthly,
        in order to have \$15,000 for a down payment in two years? \solution{\$584.07}
        \vfill
\clearpage
  \item How much money will you have when you retire if you save \$20 each month from graduation (age 22) until retirement (age 63),
        if you can average 6.6\% annual interest compounded monthly? \solution{41 years, \$50395.28}
        \vfill
  \item You start saving for a down payment on a house by depositing \$100 each month 
        into an annuity that pays 4.8\% interest, compounded monthly.
        How large a down payment can you afford in 3 years? \solution{\$3863.81}

        \vfill
  \item Hoopla Publishing Company knows its printing press is nearing the end of its life.
        They will need to purchase a new printing press for \$65,000 in 10 years.
        What payment should Hoopla make every year into a sinking fund earning
        7\% interest, compounded annually,
        in order to have the \$65,000 in ten years? \solution{\$4,704.54}
        \vfill
\vspace{-1in}
\clearpage
\item Suppose you wanted to compute how much money would be in an account earning 5\% interest compounded monthly if you deposited \$100/month for 25 years.
\begin{enumerate}
	\item How much would you have if you used $\displaystyle \frac{0.05}{12}\simeq 0.004$? \vfill
	\item How much would you have if you used $\displaystyle \frac{0.05}{12}\simeq 0.0041$? \vfill
	\item How much would you have if you used $\displaystyle \frac{0.05}{12}\simeq 0.00417$? \vfill
	\item How much would you have if you used $\displaystyle \frac{0.05}{12}\simeq 0.004167$? %\vfill
\end{enumerate}
\ifsolns  \$57,804.48 ,
 \$59,907.89 ,
 \$59,586.55 ,
 \$59,554.53 
\fi \vfill
\item Suppose you wanted to save up \$60,000 over 20 years by depositing an amount of money in the bank account earning 5\% interest compounded monthly.
\begin{enumerate}
	\item How much would you have to save every month if you used $\displaystyle \frac{0.05}{12}\simeq 0.004$? \vfill
	\item How much would you have to save every month if you used $\displaystyle \frac{0.05}{12}\simeq 0.0041$? \vfill
	\item How much would you have to save every month if you used $\displaystyle \frac{0.05}{12}\simeq 0.00417$? \vfill
	\item How much would you have to save every month if you used $\displaystyle \frac{0.05}{12}\simeq 0.004167$? %\vfill
\end{enumerate}
\ifsolns   \$149.37 ,
 \$145.30 ,
 \$145.91 ,
 \$145.97 
\fi\vfill

\end{enumerate}

%\end{enumerate}

%</WORKSHEETS>

%<*HWHEADER>
%%%%%%%%%%%%%%%%%%%%%%%%%%%%%%%%%%%%%%%%%%%%%%%%%%%%%%%%%%%%%%%%%%%%%%%%%%%%%%%%%%%%%%%%%%%%%%%%%%%%%%
\HOMEWORK
%</HWHEADER>

%<*HOMEWORK>

\begin{Fenumerate}
  \item Find the future value of each of the following ordinary annuities.
        \begin{enumerate}
          \item Deposits of \$1200 made at the end of each year    for 10 years, where 7\% annual interest is compounded annually.
                \ifsolns
                  \par \soln
                    $\$1200\dfrac{(1+\frac{.07}{1})^{10}-1}{\frac{.07}{1}} = \boxed{\$16579.74.}$
                \else \vfill \fi
          \item Deposits of \$300  made at the end of each quarter for 10 years, where 8\% annual interest is compounded quarterly.\label{prob:FVq}
                \solution*{$\$300\dfrac{(1+\frac{.08}{4})^{4\cdot 10}-1}{\frac{.08}{4}} = \boxed{\$18{,}120.59.}$} \vfill
%                \ifsolns
%                  \par \soln
%                    $\$300\dfrac{(1+\frac{.08}{4})^{4\cdot 10}-1}{\frac{.08}{4}} = \boxed{\$18120.59.}$
%                \else \vfill \fi
          \item Deposits of \$51   made at the end of each month   for 20 years, where 6\% annual interest is compounded monthly.
                \ifsolns
                  \par \soln
                    $\$51\dfrac{(1+\frac{.06}{12})^{20\cdot 12}-1}{\frac{.06}{12}} = \boxed{\$23564.09.}$
                \else \vfill \fi
          \item Deposits of \$100  made at the end of each week    for 2  years, where 8\% annual interest is compounded weekly.
                %(Assume 52 weeks in a year.)
                \solution*{$\$100\dfrac{(1+\frac{.08}{52})^{52\cdot 2}-1}{\frac{.08}{52}} = \boxed{\$11268.83.}$}
        \end{enumerate}
				\vfill
  \item Francine Spiffhaven deposits \$50 each week in a bank CD earning 4\% interest compounded weekly.
        She does this for 14 years.  
        \begin{enumerate}
          \item How much is in her bank account at the end of those 14 years? \par
                %(As usual, assume there are 52 weeks in a year.)
                \solution{%
                  $\$50\dfrac{(1+\frac{0.04}{52})^{52\cdot 14}-1}{\frac{0.04}{52}} = \boxed{\$48{,}769.22}$
                }
                \studentsoln{\$48,769.22}\vfill
          \item How many dollars of interest did she earn, in total, over those 14 years?
                \solution{%
                  She deposited a total of $\$50 \cdot 52\cdot 14 = \$36{,}400$, so the interest she earned is
                  $\$48{,}769.22 - \$36{,}400 = \boxed{\$12{,}369.22.}$
                }
                \studentsoln{\$12,369.22}\vfill
        \end{enumerate}
\hwnewpage

  \item Edgar is graduating from Wartburg College and wants to have \$20,000 ready for a down payment on a house in 5 years.
        If his investments pay him 6\% interest, compounded monthly,
        how much money should he invest every month in order to achieve his goal?
        \solution{We want $\$20{,}000 = P \dfrac{(1+\frac{0.06}{12})^{12\cdot 5}-1}{\frac{0.06}{12}}$, \\
                  so $P = \$20{,}000 \div \dfrac{(1+\frac{0.06}{12})^{12\cdot 5}-1}{\frac{0.06}{12}} = \boxed{\$286.66.}$}
        \studentsoln{\$286.66}\vfill
  \item \begin{enumerate}
          \item Joe Parsimmons said he would set up an ordinary annuity for his newborn niece Gloria
                and deposit \$100 each month, with the last payment to occur on her 18th birthday.
                The payments would earn 6\% annual interest, compounded monthly.
                How much will Gloria have on her 18th birthday?
                \ifsolns
                  \par \soln
                    $\$100\dfrac{(1+\frac{.06}{12})^{12\cdot 18}-1}{\frac{.06}{12}} = \boxed{\$38735.32.}$
                \fi
                \studentsoln{\$38,735.32}\vfill
          \item Joe's wife Susan suggested they should just deposit a lump sum of money \emph{now}
                into a bank account (earning 6\% annual interest, compounded monthly),
                so that it would grow to the same amount by Gloria's 18th birthday as you found in part (a).
                If they go with Susan's plan, how much money must the loving aunt and uncle deposit in the account now?
                \ifsolns
                  \par \soln
                    We want to find $P$ such that $\$38735.32=P\left(1+\frac{.06}{12}\right)^{12\cdot 18}$;
                    solving, we find they would need to deposit \fbox{\$13,189.79} today.
                \fi
                \studentsoln{\$13,189.79}\vfill
        \end{enumerate}
\hwnewpage
  \item Zeke Fitzhugh decides to pay \$300 at the end of each month into an ordinary annuity that pays
        8\% annual interest, compounded monthly, for five years.
        He decides to calculate the future value of this annuity at the end of five years,
        but he makes a mistake in his calculations.
        What was his mistake?  Is his answer too big or too small?
        \begin{eqnarray*}
          FV &=& P \cdot \left( \frac{(1+i)^m-1}{i}\right) \\
          FV &=& 300 \cdot \left( \frac{(1+.08)^{60}-1}{.08}\right) \\
          FV &=& 300 \cdot 1253.213296 \\
          FV &=& \$375{,}963.99
        \end{eqnarray*}
        \ifsolns
          \par \soln
            Zeke used $i=.08$, which is wrong.  The \emph{annual} interest rate is $.08$, 
            but the interest rate \emph{per month} is only $.08/12\approx .0066667$.
        \fi
        \studentsoln{His answer is too big.}
\end{Fenumerate} \ENDHOMEWORK
%%%%%%%%%%%%%%%%%%%%%%%%%%%%%%%%%%%%%%%%%%%%%%%%%%%%%%%%%%%%%%%%%%%%%%%%%%%%%%%%%%%%%%%%%%%%%%%%%%%%

\cleartooddpage  