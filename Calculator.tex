\def\calckey#1{\fbox{\raisebox{0pt}[6pt][0pt]{\footnotesize\tt #1}}}

\def\calcscreen#1{%
\begin{center}%
  \fbox{%
    \begin{minipage}{2in}
      \tt
      #1
    \end{minipage}
  }
\end{center}%
}
\def\ca{\rule{0pt}{0pt}\hfill}

\def\caret{\char`\^}

\def\rightkey{\calckey{$\blacktriangleright$}}
\def\stokey{\calckey{STO$\blacktriangleright$}}

\def\tineg{\raisebox{1pt}[0pt][0pt]{-}}

\def\sto{\ensuremath{\rightarrow}}

\chapter{Calculator Tips} \label{ch:CalculatorTips}

Chapter \ref{ch:finance} requires the use of a calculator to evaluate some rather complicated formulas.
This appendix contains some advice on how to avoid pitfalls that sometimes afflict students.
Most of the advice applies to all calculators, but some is geared specifically to Texas Instruments calculators
like the TI-84 and TI-30.

\section{Parentheses}

One of the greatest challenges to students is using parentheses correctly to help a calculator interpret a formula correctly.
For example, consider the expression
\[\frac{1+7}{2}.\]
We know from elementary school that you compute the whole numerator of the fraction before dividing by the denominator,
so this expression equals $4$.
If you type it into the calculator without parentheses, however, we get
\calcscreen{1+7/2 \\
            \ca 4.5}
That's because the calculator, following the order of operations we learned as children,
first does the division and then the addition: ${\tt 7/2} = 3.5$, so ${\tt 1+7/2}=4.5$.

We human beings can tell, by looking at $\frac{1+7}{2}$, that the $1+7$ has to be done first,
but the calculator doesn't know that unless we tell it.  So we should really type
\calcscreen{(1+7)/2 \\
            \ca 4}
One way to make this process easier, especially in long complicated formulas, is to draw in the extra parentheses on your paper before you begin typing into the calculator.
\[\mbox{Given~}\frac{1+7}{2},\mbox{~draw in~}\frac{\textcolor{red}{(}1+7\textcolor{red}{)}}{2}\mbox{~and then type~}{\tt (1+7)/2}.\]

Here are some locations that typically need extra parentheses:
\begin{itemize}
  \item The numerator of a fraction:
        \begin{center}
            $\dfrac{1+7}{2}$ becomes $\dfrac{\textcolor{red}{(}1+7\textcolor{red}{)}}{2}$. \\
          \begin{tabular}{ll}
            \textbf{Incorrect:} & \textbf{Correct:} \\
            \fbox{\begin{minipage}{2in} \tt
              1+7/2 \\
              \ca 4.5
            \end{minipage}}
            &              
            \fbox{\begin{minipage}{2in} \tt
              (1+7)/2 \\
              \ca 4
            \end{minipage}}
          \end{tabular}
        \end{center}
  \item The denominator of a fraction:
        \begin{center}
            $\dfrac{\quad8\quad}{\frac{6}{3}}$ becomes $\dfrac{\quad8\quad}{\textcolor{red}{(}\frac{6}{3}\textcolor{red}{)}}$. \\
          \begin{tabular}{ll}
            \textbf{Incorrect:} & \textbf{Correct:} \\
            \fbox{\begin{minipage}{2in} \tt
              8/6/3 \\
              \ca 0.444444
            \end{minipage}}
            &              
            \fbox{\begin{minipage}{2in} \tt
              8/(6/3) \\
              \ca 4
            \end{minipage}}
          \end{tabular}
        \end{center}
  \item Exponents:\footnote{Some more recent Texas Instrument calculators help you out by making exponents actually look like exponents.
                            On such a calculator, if you press the keys \calckey5\,\calckey{\caret}\,\calckey6\,\calckey{-}\,\calckey4\,, it will display as
                            \calcscreen{$\tt 5^{6-4}$ \\ \ca 25}
                            On this kind of calculator, if you need to type more \emph{after} the exponent, press the right arrow key \rightkey\ to get ``back down'' to the main line.
                            For example, to evaluate $5^{6-4}+7$ on such a calculator, you would press the buttons
                            \calckey5\,\calckey{\caret}\,\calckey6\,\calckey{-}\,\calckey4\,\rightkey\,\calckey{+}\,\calckey7
                            to get
                            \calcscreen{$\tt 5^{6-4}+7$ \\ \ca 32}}
        \begin{center}
            $5^{6-4}$ becomes $5^{\textcolor{red}{(}6-4\textcolor{red}{)}}$. \\
          \begin{tabular}{ll}
            \textbf{Incorrect:} & \textbf{Correct:} \\
            \fbox{\begin{minipage}{2in} \tt
              5\caret6-4 \\
              \ca 15621
            \end{minipage}}
            &              
            \fbox{\begin{minipage}{2in} \tt
              5\caret(6-4) \\
              \ca 25
            \end{minipage}}
          \end{tabular}
        \end{center}

        

\end{itemize}
Of course, one complicated expression may need all of these!
        \begin{center}
            $500\dfrac{\left(1+\frac{.07}{52}\right)^{8\cdot52}-1}{\frac{.07}{52}}$ becomes 
            $500\dfrac{\textcolor{red}{\Big(}\left(1+\frac{.07}{52}\right)^{\textcolor{red}{(}8\cdot52\textcolor{red}{)}}-1\textcolor{red}{\Big)}}{\textcolor{red}{(}\frac{.07}{52}\textcolor{red}{)}}$.  \\
          \begin{tabular}{ll}
            \textbf{Incorrect:} & \textbf{Correct:} \\
            \fbox{\begin{minipage}{2in} \tt
              500*(1+.07/52)\caret8*52-1/ \\
              .07/52 \\
              \ca 26281.04806
            \end{minipage}}
            &              
            \fbox{\begin{minipage}{2in} \tt
              500*((1+.07/52)\caret(8*52) \\
              -1)/(.07/52) \\
              \ca 278576.3860
            \end{minipage}}
          \end{tabular}
        \end{center}


\section{Rounding Errors}

We all know that $\frac{1}{3} \times 300 = 100$, 
but sometimes when working with a calculator we can get wrong answers:
\calcscreen{1/3 \\
            \ca .3333333333 \\
            .3333333333*300 \\
            \ca 99.99999999}
The calculator is not at fault here; it has not made any mistakes.
The decimal equivalent of $\rec{3}$ is $0.33333333333333333333\ldots$;
it goes on forever.  The calculator gives us as many digits as it can,
but {\tt .3333333333} is just a \emph{rounded-off approximation} of $\rec{3}$.
When we then typed {\tt .3333333333} back into the calculator and multiplied by 300,
the calculator accurately told us the answer was 99.99999999, not 100.
This is called an \emph{rounding error}.

In this example, the difference between what the calculator told us and the real answer
is only $0.00000001$, which is not terribly bad.
Often students round off long decimals to only a few decimal places,
which is okay for a final answer, but then they run into more problems:
\calcscreen{1/3 \\
            \ca .3333333333 \\
            .33*300 \\
            \ca 99}
Now we got an answer of 99 instead of 100, which seems more significant.
In the more complicated expressions of the financial math chapter,
we can end up with errors of several dollars (possibly hundreds of dollars) if we're not careful.
For example, if we need to calculate $\$600 \cdot \frac{1-(1+\frac{.07}{12})^{-12\cdot 25}}{\frac{.07}{12}}$,%$\ds \$600 \cdot \frac{1-(1+\frac{.07}{12})^{-12\cdot 25}}{\frac{.07}{12}}$,
a common student error would be this:
\calcscreen{(1-(1+.07/12)\caret(\tineg12*25))/\\
            (.07/12) \\
            \ca 141.486903386 \\
            600*141.49 \\
            \ca 84894}
Whereas this student answered \$84,894.00,
the correct answer (to the nearest penny) is \$84,892.14;
that's \$1.86 too high, enough to lose points on a test.

Thus when the calculator gives you numbers you will use again later in the problem,
\emph{never round them off}.  Always type back in \emph{all} the digits.
Only round off at the very end of the problem!

However, this can \emph{still} cause small errors (as we saw in the first example),
and it's annoying to punch {\tt 141.486903386} into the calculator.
There are at least three better solutions to the rounding error problem,
all of which involve \emph{keeping the numbers in the calculator until the end}.

\begin{itemize}
  \item One solution is to combine your whole formula into one step in the calculator.
        \begin{center}
          \begin{tabular}{ll}
            \textbf{Example 1:} & \textbf{Example 2:} \\
            \fbox{\begin{minipage}[t]{2in} \tt
              (1/3)*300 \\
              \ca 100
            \end{minipage}}
            &              
            \fbox{\begin{minipage}[t]{2in} \tt
              600*(1-(1+.07/12)\caret(\tineg12*\\
              25))/(.07/12) \\
              \ca 84892.1420313
            \end{minipage}}
          \end{tabular}
        \end{center}
        The downside of this approach is that 
        writing the formula all in one line can be long,
        error-prone, and just plain unpleasant.

  \item An alternative is to break up your calculation into steps using {\tt Ans}.
        Most calculators have a feature to remember the most recent answer
        they computed; on a TI machine, this is represented by the symbol {\tt Ans}.
        The {\tt Ans} automatically appears if you start a line by pressing an operation like \calckey{+} or \calckey{$\div$}\,,
        or you can also get it by pressing the \calckey{2ND} key 
        and then the \calckey{(-)} key.
        \begin{center}
          \begin{tabular}{ll}
            \textbf{Example 1:} & \textbf{Example 2:} \\
            \fbox{\begin{minipage}[t]{2in} \tt
              1/3 \\
              \ca 0.3333333333 \\
              Ans*300 \\
              \ca 100
            \end{minipage}}
            &              
            \fbox{\begin{minipage}[t]{2in} \tt
              (1-(1+.07/12)\caret(\tineg12*25))/\\
              (.07/12) \\
              \ca 141.486903386 \\
              600*Ans\\
              \ca 84892.1420313
            \end{minipage}}
          \end{tabular}
        \end{center}
        In Example 1, the keystrokes used were 
        \calckey1\,\calckey/\,\calckey3\,\calckey{ENTER}
        and then
        \calckey*\,\calckey3\,\calckey0\,\calckey0\,\calckey{ENTER}\,;
        the {\tt Ans} was automatically generated by the calculator.
        In Example 2, 
        the keys punched in the second line were
        \calckey6\,\calckey0\,\calckey0\,\calckey*\,\calckey{2ND}\,\calckey{(-)}\,\calckey{ENTER}\,.

  \item The calculator's {\tt Ans} ability remembers just one number,
        the result of the most recent calculation.
        There is also a more powerful way to use the calculator's memory,
        using \emph{variables}, which are letters that stand for numbers.

        Most keys on your TI calculator have a letter printed above them and to the right;
        on a TI-84 Plus Silver Edition calculator, for example, one key looks like this: \\
        \centerline{%
          \tt\begin{tabular}{c}
               \tiny TEST \hfill \footnotesize A \\
               \fbox{MATH}
             \end{tabular}}
        \medskip
        
        To get the letter, you first press the \calckey{ALPHA} key,
        and then the key with the desired letter;
        for example, \calckey{ALPHA}\,\calckey{MATH} produces an ``{\tt A}'' on the screen.

        You can make the calculator remember any number at all 
        by \emph{storing} it to a variable, whose name is a letter.
        You use the \stokey\ key, 
        which writes a little ``$\rightarrow$'' symbol on the screen.  
        Then you choose which letter you want to represent your number,
        and press \calckey{ENTER}.
        The first line of this example was created on my TI-84 Plus Silver Edition by pressing
        \calckey5\,\stokey\,\calckey{ALPHA}\,\calckey{A}\,\calckey{ENTER}\,.
        \calcscreen{4+1$\rightarrow$A \\
                    \ca 5 \\
                    5-2$\rightarrow$B \\
                    \ca 3 \\
                    A*B \\
                    \ca 15}
        The use of variables can be immensely helpful, 
        especially if you are doing lots of calculations with the same numbers;
        they also sometimes mean you need fewer parentheses.
        Take a look at our two examples one last time.
        \begin{center}
          \begin{tabular}{ll}
            \textbf{Example 1:} & \textbf{Example 2:} \\
            \fbox{\begin{minipage}[t]{2in} \tt
              1/3\sto F \\
              \ca 0.3333333333 \\
              F*300 \\
              \ca 100
            \end{minipage}}
            &              
            \fbox{\begin{minipage}[t]{2in} \tt
              .07/12\sto I \\
              \ca .005833333333 \\
              12*25\sto M \\
              \ca 300 \\
              (1-(1+I)\caret-M)/I\sto A \\
              \ca 141.486903386 \\
              600*A\\
              \ca 84892.1420313
            \end{minipage}}
          \end{tabular}
        \end{center}
\end{itemize}

\clearpage
\section{Subtraction versus Negatives}

On Texas Instruments calculators, there are two keys with horizontal lines.
The main \calckey{-} key is for \emph{subtraction}, like $5-4$.
When you type in a \emph{negative number} like $-6$, however, 
you must use the smaller \calckey{(-)} key, usually located in the bottom row
of your calculator.
If you use the wrong button, you will probably see a screen like this:
\calcscreen{ERR:SYNTAX \\
            \raisebox{-2pt}{\makebox[0pt][l]{\rule{12pt}{11pt}}}\textcolor{white}{1:}Quit \\
            2:Goto}

\section{Correcting Mistakes}

At some point in this course you will probably type in a long, complex expression wrong,
and you'll have to fix it.
Instead of retyping the whole thing, you might be able to just edit your earlier attempt!
One of the following may work on your calculator:
\begin{itemize}
  \item pressing the up arrow key \fbox{$\blacktriangle$} to scroll up to the earlier line
  \item pressing \fbox{\tt \footnotesize 2ND}\,\fbox{\footnotesize\tt ENTER} 
        to access the latest ``{\tt\footnotesize ENTRY}''
\end{itemize}
If you do this, you might need to insert new symbols, such as parentheses, into your earlier wrong entry.
Be sure to change into insert mode (``{\tt \footnotesize INS}'')
by pressing \fbox{\tt \footnotesize 2ND}\,\fbox{\tt\footnotesize DEL}\,;
this will let you insert new characters.
Unfortunately, the calculator will switch back to normal ``overtype'' mode as soon as you press an arrow key to move to another location,
so you'll need to press \fbox{\tt \footnotesize 2ND}\,\fbox{\tt\footnotesize DEL} again.
