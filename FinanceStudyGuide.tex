\section{Study Guide} \label{sec:FinanceStudyGuide}

To prepare for the exam over this chapter,
you should the in-class worksheets and homework.
Be ready to do the kind of problems you faced on the homework.

As a general guide, I recommend reviewing the following topics.

\begin{enumerate}
  \item Calculate compound interest; use this to determine the future savings from a \emph{one-time deposit}.
  \item Find interest per period.
  \item Calculate the future value of an annuity; use this to determine future savings when making \emph{regular deposits}.
  %\item Compare two savings accounts using the \defnstyle{effective annual yield}
  \item Calculate (a) how much you can save for the future by making a series of regular deposits,
        and (b) how much you must deposit regularly to have a certain amount in the future.
  \item Calculate how much money you need now in order to fund a series of future payments.
  \item Calculate the regular payment for a loan;
        conversely, given the regular payment, calculate how much you can borrow.
  \item Calculate the equity held in a house, given information about the home's value and the mortgage.
  \item Calculate mortgage costs involving down payments and closing costs.
  \item Calculate mortgage costs involving taxes and insurance (PITI).
  %\item Given information like in problem \ref{prob:Zeke}, calculate someone's taxable income.
  %\item Given someone's taxable income
  %      and the tax rates as in problem \ref{prob:Zeke}, 
  %      estimate someone's income tax.
  %\item Understand the difference between tax deductions and tax credits.
  %      Be able to use tax deductions and credits in figuring someone's income tax.
  %\item Understand the difference between refundable and non-refundable credits,
  %      and do examples.
  %
  %\item Estimate the price of items in the future due to inflation.
  %\item Calculate the price of items in the past due to inflation.
\end{enumerate}

\clearpage
\ifsolns
\section{Review Worksheet}
\begin{enumerate}
	\item 	You take out a subsidized student loan at 4.66\% compounded monthly.  You have a monthly discretionary income  of \$ P/month of which you can put up to 20\% towards paying off your loan.  To calculate P, take your current age divided by 3 times 100.  What is the largest loan you can get under each plan?

\textbf{Standard Repayment}
With the standard plan, you'll pay a fixed amount each month until your loans are paid in full. Your monthly payments will be at least \$50, and you'll have up to 10 years to repay your loans.
The standard plan is good for you if you can handle higher monthly payments because you'll repay your loans more quickly. Your monthly payment under the standard plan may be higher than it would be under the other plans because your loans will be repaid in the shortest time. For the same reason -- the 10-year limit on repayment -- you may pay the least interest.

\textbf{Extended Repayment}
To be eligible for the extended plan, you must have more than \$30,000 in Direct Loan debt and you must not have an outstanding balance on a Direct Loan as of October 7, 1998. Under the extended plan you have 25 years for repayment.

\item	Typically, most lenders suggest that you spend no more than 28\% of your monthly income on a mortgage.  Today, home loans for a 30 year fixed-rate mortgage are running at approximately 4\%.  What is the largest loan you can afford on your monthly income of \$P?\label{EPpart2}
\item	You want to take 5 years to save up a 20\% down payment for your house.  What is the value of the house if you take out the loan you found in part \ref{EPpart2} with a 20\% down payment?  How much would you have to put aside monthly to save up that amount?  What percentage of your monthly income of \$P is that?
\item	If you are also spending 28\% of your monthly income on rent, how much money do you have to live on during the time you are paying off your loan and saving up for your house?
\end{enumerate}
\begin{center}
	\begin{tabular}{llllll}
Age & 18 & 19 & 20 & 21 & 22\\ \hline 
income &  \$600.00  &  \$633.33  &  \$666.67  &  \$700.00  &  \$733.33 \\ \hline 
10 year &  \$11,493.00  &  \$12,131.50  &  \$12,770.00  &  \$13,408.50  &  \$14,047.00 \\ \hline 
25 year &  \$21,240.70  &  \$22,420.74  &  \$23,600.78  &  \$24,780.82  &  \$25,960.86 \\ \hline 
mortgage payment &  \$168.00  &  \$177.33  &  \$186.67  &  \$196.00  &  \$205.33 \\ \hline 
Mortgage &  \$35,189.49  &  \$37,144.46  &  \$39,099.43  &  \$41,054.40  &  \$43,009.37 \\ \hline 
House Value &  \$43,986.86  &  \$46,430.57  &  \$48,874.29  &  \$51,318.00  &  \$53,761.72 \\ \hline 
20 percent &  \$8,797.37  &  \$9,286.11  &  \$9,774.86  &  \$10,263.60  &  \$10,752.34 \\ \hline 
monthly savings &  \$132.69  &  \$140.06  &  \$147.44  &  \$154.81  &  \$162.18 \\ \hline 
percentage & 22\% & 22\% & 22\% & 22\% & 22\%\\ \hline 
Live on &  \$179.31  &  \$189.27  &  \$199.23  &  \$209.19  &  \$219.15 \\ \hline 
percentage & 30\% & 30\% & 30\% & 30\% & 30\%\\ \hline 
\end{tabular}

\end{center}
\fi
